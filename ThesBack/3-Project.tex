\chapter{Type IIA on the quintic}

%\section{Motivation}
%
%%\todoin{Why the quintic?}
%%
%%\todoin{Why study SLags?}
%
%\section{SLags on Fermat's quintic}
%In this section, we describe Fermat's quintic and study the properties of SLag three-cycles 
%defined on this manifold.
%
%\subsection{Fermat's quintic}
%%The construction of generic Calabi-Yau manifolds is a 
%%Chow's theorem asserts  that any $(n-r)$-dimensional submanifold of $\mathbb{CP}^n$ can be realized as the zero locus of $r$-homogeneous polynomial equations.
%%\todoin{Motivate better}
%The complex projective space $\mathbb{CP}^n$ is defined considering the complex space minus the origin $\mathbb{C}^{n+1}\backslash\qty{0}$
%and establishing the equivalence relation $( z_0,\ldots, z_{n})\sim(\lambda z_0,\ldots,\lambda z_{n})$, $\lambda\in\mathbb C$.
%%To emphasize that $z_i$ are homogeneous coordinates, they are sometimes denoted as $\qty[z_0:\ldots:z_{n}]$.
%The local (non-homogeneous) coordinates $\xi^i$ of a $j$-patch where $z_j\neq0$ are obtained  in
%terms of the homogeneous coordinates $z_i$ by choosing $\lambda=1/z_j$ so
%\begin{equation}
%  (\xi^1_j,\ldots,\xi^n_j)=\qty(\frac{z_0}{z_j},\ldots,\frac{z_{j-1}}{z_j},\frac{z_{j+1}}{z_j},\ldots,\frac{z_{n}}{z_j}).
%\end{equation}
%%It can also be shown that $\mathbb{CP}^n$ is a Kähler manifold.
%%$PGL$
%
%The complex projective space allows us to obtain lower dimensional manifolds as subspaces of $\mathbb{CP}^n$.
%Indeed, Chow's theorem states that any submanifold of complex dimension $n-r$ of $\mathbb{CP}^n$ can be realized as the zero locus of $r$-homogeneous polynomial equations.
%An homogeneous polynomial $P$ of degree $d$ satisfies
%\begin{equation}
%  P(\lambda z_1,\ldots,\lambda z_n)=\lambda^d P(z_1,\ldots,z_n).
%\end{equation}
%We are interested in calculating the number of inequivalent $(n-1)$-dimensional submanifolds of $\mathbb{CP}^n$ that can be defined
%through a polynomial equation of degree $d$.
%The number of independent monomials of degree $d$ in $n+1$ variables is given by the binomial coefficient
%\begin{equation}
%  \# \text{ indep. monomials}={{d+(n+1)-1}\choose{(n+1)-1}}
%\end{equation}
%Not all of these lead to different manifolds, since some of them can be related through coordinate transformations, 
%which belong to complex general linear group $GL(\mathbb C)_{n+1}$.
%Thus, the number of possible submanifolds is given by
%\begin{equation}
%  \# \text{ submanifolds } \mathbb{CP}^n = \# \text{ indep. monomials}-\# \text{ components }GL(\mathbb C)_{n+1}
%\end{equation}
%Among of the ${{9}\choose{4}}-5^5=101$ possible submanifolds of $\mathbb{CP}^4$ defined by a quintic polynomial, lies
%Fermat's quintic threefold defined by the polynomial $P_5$ as
%\begin{equation}
%  %P_4(z)& = z_0^5+z_1^5+z_2^5+z_3^5=0\\
%  P_5(z) = z_0^5+z_1^5+z_2^5+z_3^5+z_4^5=0.
%\end{equation}
%
%\subsection{Construction of SLags}
%A special-Lagrangian three-cycle is constructed by introducing an anti-holomorphic involution.
%In the case of $\mathbb{CP}^4$, there is only one consistent anti-holomorphic involution,
%which acts on the homogeneous coordinates as $\mathcal R: z_i \to\bar z_i$.
%A particular instance of SLag three-cycle is defined as the fixed loci under $\mathcal R$.
%In the case of Fermat's quintic leads it to the following SLag
%\begin{equation}
%  \qty{\qty[x_0:x_1:x_2:x_3:x_4]\in  \mathbb{RP}^4| x_0^5+x_1^5+x_2^5+x_3^5+x_4^5=0}.
%\end{equation}
%This subspace is topologically equivalent to $\mathbb{RP}^3$, which can be noticed through the
%homeomorphism from $\mathbb{RP}^3$ to the SLag:
%\begin{equation}
%(u_0,u_1,u_3) \to (u_0,u_1,u_2,u_3, -(u_0^5+u_1^5+u_2^5+u_3^5)^{1/5}).
%\end{equation}
%
%In order to prove that the obtained subspace is indeed special-Lagrangian we must
%verify that the Kähler form vanishes at the SLag and that it is calibrated with respect to the
%same three-form as the $\Omega_3$ of the quintic.
%\begin{itemize}
%  \item The Kähler form $k$ transforms under the anti-holomorphic involution as $k\to -k$, 
%    but since the pull-back of $k$ onto the three-dimensional subspace must be real, 
%    the only possibility is $k\rvert_{\mathrm{SLag}}=0$
%
%  \item 
%The holomorphic three-form in a non-homogeneous patch where $x_0\neq0$ is
%\begin{equation}
%  \Omega_3=\frac{4}{2\pi i}\int \frac{x_0 dx_1\wedge dx_2 \wedge dx_3 \wedge dx_4}{P_5(x_i)}.
%  \label{eq:3form}
%\end{equation}
%If we interpret $x_4$ as a function of $P_5$ and integrate over a loop around $P_5=0$
%\begin{equation}
%  \Omega_3=\frac{x_0 dx_1\wedge dx_2 \wedge dx_3}{x_4^4}
%  =\frac{dy_1\wedge dy_2\wedge dy_3}{y_4^4}
%\end{equation}
%where we have defined the local coordinates $y_i=x_i/x_0$.
%By taking the norm of the previous equation, we relate $\Omega_3$ to the determinant of the 
%six-dimensional metric
%\todoin{Why metric appears?}
%\begin{equation}
%  ||\Omega_3||^2 = \frac{1}{\det g |y_4|^8}.
%\end{equation}
%Since $\Omega_3$ is covariantly constant, $||\Omega_3||$ must also be proportional to a constant defined as
%\begin{equation}
%  ||\Omega_3||^2 = 8e^{2\kappa}.
%\end{equation}
%The pull-back of the six-dimensional metric onto the three-cycle is
%\begin{equation}
%  h_{\alpha\beta}=2\partial_\alpha X^i \partial_\beta X^{\bar j}g_{i\bar j}
%\end{equation}
%So the volume form of the SLag three-cycle coincides with $\Omega_3$
%\begin{equation}
%  e^k \sqrt{\det h_{ab}}d\sigma^1\wedge d\sigma^2\wedge d\sigma^3
%  =e^k |\det \partial y| \frac{e^{-\kappa}}{|y_4|^4}d\sigma^1\wedge d\sigma^2\wedge d\sigma^3=\Omega_3
%\end{equation}
%\todoin{Why lhs has that form?}
%\end{itemize}
%
%It is possible to define different SLags by exploiting the $\mathbb Z_5^4$ symmetry of the quintic,
%which leads to a whole family of SLags
%\begin{equation}
%  \ket{0,k_1,k_2,k_3,k_4} = {\qty[x_0:\omega^{k_1} x_1:\omega^{k_2} x_2:\omega^{k_3} x_3:\omega^{k_4} x_4]}
%\end{equation}
%where $\omega=e^{i\frac{2\pi}{5}}$. 
%The freedom to choose of $k_i\in \mathbb Z_5$ leads to $5^4=625$ SLag three-cycles, but not all of them are calibrated
%with respect to the same three-form as O$6$-plane $\ket{0,0,0,0,0}$.
%%$\sum_{i} k_i^4=0 \text{ mod 5}$
%Concretely, only $125$ SLags are calibrated with respect to the same three-form as the O$6$-plane.
%
%\subsection{Moduli space of SLags}
%
%\todoin{Not sure if the reasoning in this section is too sloppy.}
%We are not only interested in SLags themselves, but also in their moduli space.
%The reason is that deformations of SLag three-cycles codify the information of the open string 
%excitations of the D$6$-branes that wrap around them.
%In particular, they determine de vev of the scalar field excitations and thus the position of the D6-branes
%and whether they are rigid configurations or not.
%
%A deformation of a reference SLag three-cycle $\Pi_3$ into another three-cycle (in the same homology class as the original SLag)  can be parameterized by a normal vector field, which belongs to the normal bundle of $\Pi_3$.  
%This deformation normal vector field can be written in a basis $s^i$ as
%\begin{equation}
%  X=\phi_i s^i
%\end{equation}
%where $\phi_i$ correspond to the scalar fields defined on the D6-brane worldvolume theory.
%The Kähler form introduces an isomorphism between the normal bundle of $\Pi_3$ and the cotangent bundle of $\Pi_3$.
%According to McLean's theorem, harmonic one-forms are associated to a deformation vector field.
%Recalling that there is a single harmonic form in each cohomology class, we associate to every $s^i$ a one-form $\xi_i\in H^1(\Pi)$.
%As a consequence, the moduli space of SLags is identified with $H^1(\Pi_3)$.
%
%A gauge potential $A$ on the D$6$-brane can be expanded in a basis of one-forms $\xi_i$ as $A=a^i \xi_i$.
%In order to fill in the multiplets of the spectrum arising from open string excitation of D6-branes,
%we must arrange the $a^i$ and $\phi_i$ into a complex scalar field
%\begin{equation}
% \Phi^i = \phi_i + ia^i. 
%\end{equation}
%
%We can now apply the previous reasoning to SLag three-cycles defined on Fermat's quintic.
%Since $H^1(\mathbb{RP}^3)\simeq H_1(\mathbb{RP}^3)\simeq\mathbb Z_2$, there are no continuous deformations of the
%SLags and the SLags are guaranteed to remain rigid.

\subsection{Intersection numbers}
We now study the intersection numbers between different SLags, which are used to determine the number 
of chiral fermions associated to the intersecting D6-branes.

First of all, we choose a coordinate patch of $\mathbb{CP}^4$ where $x_0=1$ and take one of the SLags to be $\ket{0,0,0,0,0}$
and the other SLag $\ket{0,k_1,k_2,k_3,k_4}$.
The latter SLag can  be  obtained by applying successive rotations $\omega_k$ associated to the $\mathbb Z_5^4$ symmetry
as $\ket{0,k_1,k_2,k_3,k_4}=\omega^{k_1}_1\omega^{k_2}_2\omega^{k_3}_3\omega^{k_4}_4\ket{0,0,0,0,0}$.
The intersection numbers in this patch are:
\begin{itemize}
  \item If $\omega_i^{k_i}\neq 1$ for all $i$, there are no intersections in this patch. 
    The justification is that the coordinates of one of the SLags $\qty[1:x_1:x_2:x_3:x_4]$ where $x_i\in\mathbb R$ take only real values,
    while some of the coordinates of the SLag $[1:\omega^{k_1}x_1:\omega^{k_2}x_2:\omega^{k_3}x_3:\omega^{k_4}x_4]$ take
    always complex numbers.

  \item If $\omega_i^{k_i}= 1$ for only one $i$, which we set to be $i=4$, the intersection 
    is at a single point.
    In particular, the intersection of $\qty[1:x_1:x_2:x_3:x_4]$ with $[1:\omega^{k_1}x_1:\omega^{k_2}x_2:\omega^{k_3}x_3:\omega^{k_4}x_4]$
    is located at $(1,0,0,0,x_4)$. 
    When we restrict this line in $\mathbb{CP}^4$ to the quintic hypersurface, the intersection reduces to the point $(1,0,0,0,-1)$.
    The signature of the intersection is $\mathrm{sgn}(\Im \omega^{k_1}\Im\omega^{k_2}\Im\omega^{k_3})$. 

  \item If $\omega_i^{k_i}=1$ for at least two different values of $i$, we have to deform one of the SLags. 
    The deformation must be normal to both SLags and through McLean's theorem  we identify the normal bundle of the intersection
    with its tangent space.
    Then, the intersection number is the number of zeros of the non-trivial vector fields 
    that can be defined on the intersection locus.
    \begin{itemize}
  \item
    If $\omega_i^{k_i}= 1$ for two $i$'s, the intersection is $\qty[1:0:0:x_3:x_4]$, which is homeomorphic to $\mathbb{RP}^1$ and the circle $S^1$.
    We can form a nowhere vanishing vector field on the circle, embedded in $\mathbb R^2$, associating to 
    every point  $(\sin\phi,\cos\phi)$ the vector $X(\phi)=-\sin\phi\partial_x +\cos\phi\partial_y$.
    Thus, the intersection number is zero $I_{\mathbb{RP}^1}=0$.
  \item 
    If $\omega_i^{k_i}= 1$ for three $i$'s, the intersection is $[1:0:0:x_3:x_4]$, which is homeomorphic to the $\mathbb{RP}^2$.
    The double cover of $\mathbb{RP}^2$ is the two-sphere $S^2$, which we consider embedded in $\mathbb R^3$.
    We define the vector field
    \begin{align}
      X_\theta& =\cos\phi\cos\theta \partial_x+\cos\theta\sin\phi\partial_y-\sin\theta\partial_z\\
      X_\phi&=\sin\theta(-\sin\phi\partial_x+\cos\phi\partial_y)
    \end{align}
    where $X_\theta$ vanishes at the poles, while $X_\phi$ is not (uniquely) defined.
    These two zeros reduce to a single zero when we relate antipodal points of $S^2$ to a single point
    of $\mathbb{RP}^2$, so the intersection number is one $I_{\mathbb{RP}^2}=1$.
    The orientation of the intersection is $\mathrm{sgn}(\Im \omega^{k_j})$ $j\neq i$.
  \item 
    If $\omega^{k_i}=1$ for all $i$'s, the intersection is $\mathbb{RP}^3$, whose double cover is $S^3$.
    Since the vector fields on $S^3$ have no zeros, there are no intersections $I_{\mathbb{RP}^3}=0$.
    \end{itemize}
\end{itemize}

The total intersection number is obtained by repeating this procedure in all patches
and then adding all the obtained intersection numbers.
It can be shown that the total intersection number of intersections of
$\ket{0,0,0,0,0}$ with $\ket{0,k_1,k_2,k_3,k_4}$ that preserve supersymmetry ($\sum_i k_i=1$) is zero. 
%the coefficient of the monomial $g_1^{k_1}g_2^{k_2}g_3^{k_3}g_4^{k_4}$ in
%\begin{equation}
%  \prod_{i=0}^4 (g_i+g_i^2-g_i^3-g_i^4)
%\end{equation}


\subsection{Volumes}
We now try to compute the volume of SLag three-cycles, as they carry the information of the four-dimensional
gauge theory supported by the wrapping D$6$-branes, such as the gauge coupling.
It suffices to determine the volume of $\ket{0,0,0,0,0}$, since a generic SLag can be obtained through $\mathbb Z_4$ rotations.
The holomorphic three-form on a patch $x_0\neq0$ in terms of the local coordinates $y_i$ is
\begin{equation}
  \Omega_3=\frac{dy_1\wedge dy_2\wedge dy_3}{5y_4^4}
\end{equation}
The volume of the SLag is obtained integrating $\Omega_3$ over the three-cycle
\begin{equation}
  \mathrm{Vol}=\int_{\mathbb{RP}^3}\frac{dy_1\wedge dy_2\wedge dy_3}{5 y_4}=
\int \frac{dy_1\wedge dy_2\wedge dy_3}{5(1+y_1^4+y_2^4+y_3^4)^{4/5}}
\label{eq:int}
\end{equation}
It is not evident how to compute this integral, due to the non-trivial integration domain.

We can relate the volume of $\ket{0,0,0,0,0}$ to the volume of an associated three-sphere as follows.
The SLag three-cycle is homeomorphic to $\mathbb{RP}^3$, which in turn is diffeomorphic to the
the rotation group $SO(3)$. 
Considering that we there is a map from $S^3$ with antipodal points identified onto $SO(3)$,
we conclude that the volume of a SLag is half the volume of a corresponding three-sphere on the quintic.
In practice, we can make no further progress through this approach, since we do not know
the Calabi-Yau metric of the quintic.

\todoin{To complete}
%We come back to the integral \eqref{eq:int} where 
%the integration domain is given by the points $x_1,x_2,x_3 \in \mathbb R$ where $-(1+x_1^5+x_2^5+x_3^5)^{4/5}$ is real.
%
%integrate over the positive octant $0<x_i<\infty$
%Computed analytically
%\begin{equation}
%  \mathrm{Vol}=\frac{\Gamma\qty(\frac{1}{5})\Gamma\qty(\frac{3}{10})\Gamma\qty(\frac{11}{10})\Gamma\qty(\frac{6}{5})}{5 \times 2^{1/5}\pi}
%\end{equation}
%
%\subsection{SM on the quintic}

%\section{SLags on the deformed quintic}
%
%\subsection{Deformations of the quintic}
%
%A generalization of Fermat's quintic consists in considering hypersurfaces defined by a generic polynomial of degree five
%\begin{equation}
%  P_5(z)=\sum_{n_0+n_1+n_2+n_3+n_4=5} a_{n_0 n_1 n_2 n_3 n_4} z_0^{n_0}z_1^{n_1}z_2^{n_2}z_3^{n_3}z_4^{n_4}=0
%\end{equation}
%This construction reduces to Fermat's quintic when $a_{50000}=a_{50000}=a_{50000}=a_{50000}=a_{50000}=1$ and all other coefficients are zero.
%A well-know example is
%\begin{equation}
%  z_0^5+z_1^5+z_2^5+z_3^5+z_4^5- 5\psi z_0z_1z_2z_3z_4=0
%\end{equation}
% where the parameter $\psi$ can take three values of particular relevance
%\begin{itemize}
%  \item $\psi=1$: the Fermat (or Gepner point).
%  \item $\psi=0$: the conifold point.
%  \item $\psi=\infty$: the large complex structure limit.
%\end{itemize}
%
%We can deform Fermat's quintic by adding a monomial to the defining polynomial.
%The possible monomials are of the type:
%\begin{enumerate}
%  \item $z_0z_1z_2z_3z_4$, $1$ deformation.
%  \item $z_i z_j z_k(z_l)^2$, ${5}\choose{3}$${2}\choose{1}$$=20$ deformations. 
%  \item $z_i(z_j)^k(z_l)^2$, ${5}\choose{1}$${4}\choose{2}$$=30$ deformations.
%  \item $z_i z_j(z_k)^3$, ${5}\choose{2}$${3}\choose{1}$$=30$ deformations.
%  \item $(z_i)^2(z_j)^3$, ${5}\choose{1}$${4}\choose{1}$$=20$ deformations.
%  \item $z_i(z_j)^4$, ${5}\choose{3}$${2}\choose{1}$$=20$ deformations.
%\end{enumerate}
%In total there are $126$ deformations, but they are not all independent, since they 
%can be related through coordinate transformations.
%Subtracting  $25$ we $101$ deformation parameters.
%This is precisely the Hodge number $h_{2,1}=101$, since the defining homogeneous polynomial  
%represents a particular choice of the complex structure, which is given by 
%harmonic $(2,1)$-form in $H^{2,1}(X,\mathbb C)$.
%The deformations of Kähler structure is given by the $h^{1,1}$ and is precisely one.
%
%Fermat's quintic enjoys the $\mathbb Z_5^4$ symmetry acting on the homogeneous coordinates 
%\begin{equation}
%  z_i \to \omega^k_i z_i, \qquad \omega_i=e^{i\frac{2\pi}{5}}
%\end{equation}
%which can be broken into a smaller subgroup when turning on deformations.
%We should also examine whether deformations introduce any singularities, which
%are defined as the points 
%\begin{equation}
%  P_5(X)=0, \qquad dP_5=0
%\end{equation}
%The nature of the singularity is obtained by evaluating the Hessian at the singularity.
%That is calculating $d^2 P_5$ in a local coordinate patch, where one of the homogeneous coordinate is non-zero.
%\todoin{Complete}
%
%We proceed to examine the singularities and associated symmetry subgroups of the deformations.
%\begin{enumerate}
%  \item No deformations.
%
%    $z_0^5+z_1^5+z_2^5+z_3^5+z_4^5=0$ The only point where $dP_5=0$ is  $(0,0,0,0,0)$, which 
%    is not part of $\mathbb{CP}^4$, so 
%
%  \item $5\psi z_0z_1z_2z_3z_4$
%
%    $z_0^5+z_1^5+z_2^5+z_3^5+z_4^5- 5\psi z_0z_1z_2z_3z_4=0$
%
%    The singularities are located at:
%    \begin{equation}
%      z_i^5=\psi x_0x_1x_2x_3x_4
%    \end{equation}
%    \begin{equation}
%      \Pi z_i^5=\psi^5 ( x_0x_1x_2x_3x_4 )^5
%    \end{equation}
%    so in order to obtain a singular point, $\psi^5=1$.
%    Taking $\psi=1$, there a single singular point $(1,1,1,1,1)$ which is a node, since the Hessian doesn't vanish.
%    Locally, the node can be recast into a conifold singularity.
%
%  \item $5\psi z_0z_1z_2(z_3)^2$
%
%    $z_0^5+z_1^5+z_2^5+z_3^5+z_4^5- 5\psi z_0z_1z_2(z_3)^2=0$
%
%    Singularities in the coordinate patch where $z_3=1$
%    \begin{equation}
%      (\frac{1}{\sqrt[5]{2}}, \frac{1}{\sqrt[5]{2}}, \frac{1}{\sqrt[5]{2}}, 1,0) 
%    \end{equation}
%    $\psi=\frac{1}{\sqrt[5]{2}}$
%
%  \item $5\psi z_0(z_1)^2(z_2)^2$
%
%    $z_0^5+z_1^5+z_2^5+z_3^5+z_4^5- 5\psi z_0z_1z_2(z_3)^2=0$
%
%    \begin{equation}
%      (1, \sqrt[5]{2}, \sqrt[5]{2}, 0, 0)
%    \end{equation}
%    $\psi=\frac{1}{\sqrt[5]{2^4}}$
%
%  \item $5\psi z_0z_1(z_2)^3$
%
%    \begin{equation}
%      (\sqrt[5]{3}, \sqrt[5]{3},1, 0, 0)
%    \end{equation}
%    $\psi=\frac{1}{\sqrt[5]{3^3}}$
%
%  \item $5\psi (z_0)^2(z_1)^3$
%
%    \begin{equation}
%      (1, \sqrt[5]{\frac{3}{2}}, 0, 0, 0)
%    \end{equation}
%    $\psi=\frac{1}{\sqrt[5]{3^3}}$
%
%  \item $5\psi z_0(z_1)^4$
%
%    \begin{equation}
%      (1, \sqrt[5]{4}, 0, 0, 0)
%    \end{equation}
%    $\psi=\frac{1}{\sqrt[5]{4^4}}$
%
%    \todoin{Check values, add words}
%\end{enumerate}
%
%\subsection{SLags on the deformed quintic}
%\todoin{Generalities. Special case. Generalization}
%
%
%%\todoin{Deformation classification}
%%
%%\todoin{Coordinate redefinition freedom}
%%
%%As an example, we take as deformation $-5\phi z_1 z_2 z_3 z_4 z_5$
%%
%%\todoin{Change of variables to study geometry of the singularity}
%%
%%In order to determine the geometry near the singularity, we make the following change of 
%%variables
%%\begin{equation}
%%  \begin{aligned}
%%  x_1 &= 1 + y_1/\sqrt{10} + y_2/5 + y_4/\sqrt{50}\\
%%  x_2 &= 1 + y_1/\sqrt{10} - y_2/5 + y_4/\sqrt{50}\\
%%  x_3 &= 1 + y_1/\sqrt{10} + y_3/5 - y_4/\sqrt{50}\\
%%  x_4 &= 1 + y_1/\sqrt{10} - y_3/5 - y_4/\sqrt{50}
%%  \end{aligned}
%%\end{equation}
%%
%%In these coordinates, the quintic becomes
%%
%%\begin{equation}
%%  5(\psi -1 ) =  y_1^2 + y_2^2 + y_3^2 + y_4^2 + O( \psi -1 )
%%\end{equation}
