\chapter{Introduction}

In order to connect String Theory with the observable world, 
we would desire to obtain the Standard Model as its low-energy effective theory.
Although there is still a lot of work to do before reaching this goal,
a promising mechanism is based on Calabi-Yau orientifold compactifications of Type IIA string theory.
Most of the work has been done on toroidal orientifold compactifications, 
but from the perspective of model building, it is always useful to extend the compactifications backgrounds to smooth Calabi-Yau's.
In this thesis we study a well-know Calabi-Yau manifold, Fermat's quintic and
its potential usefulness with respect to Type IIA model building with intersecting D6-branes.

The relevant properties of the quintic that we try to describe for the 
purpose of model building are the special-Lagrangians and their deformations,
their intersection numbers and their volumes.
We also try to apply a similar treatment to the deformations of Fermat's quintic,
will are less understood.
%well studied example Fermat's quintic and try to a similar treatment of its deformations

The outline of this thesis is as follows.
In chapter 1 we describe type IIA string theory and D6-branes in flat space, following mostly the textbook \cite{Ibanez}. 
In chapter 2 we review type IIA compactifications, in particular, Calabi-Yau orientifolds, with the aid of differential geometry tools.
This chapter is based on the textbooks \cite{Ibanez}\cite{GSW1}\cite{GSW2}.
In chapter 3 we study the quintic and its implications to model building.
Finally, in chapter 4 we summarize and discuss some additional aspects which ought to be investigated in the near future.
