\chapter{Type IIA on the quintic}

\section{Quintic threefold motivation}

\todoin{Why the quintic?}

\todoin{Why study sLags?}

\section{sLags on Fermat's quintic}

\subsection{SM on the quintic}

\section{sLags on the deformed quintic}

\subsection{Quintic deformations}
The consider the possible deformations of Fermat's quintic.
There should be $101$ independent deformations, since they correspond to different complex 
structures and are given be the Hodge number $h_{21}=101$.

We can add terms of the following type to the quintic

\begin{equation}
  x_i^5, x_i^4 x_j^1, x_i^3 x_j^2, x_i^3 x_j x_k, x_i^2 x_j x_k x_l, x_1 x_2 x_3 x_4 x_5
\end{equation}

Not all of these terms are independent, since a coordinate redefiniton $GL(5,\mathbb C)$.

\todoin{Deformation classification}

\todoin{Coordinate redefinition freedom}


\subsection{Singularities}

As an example, we take as deformation $-5\phi z_1 z_2 z_3 z_4 z_5$


\todoin{Change of variables to study geometry of the singularity}


In order to determine the geometry near the singularity, we make the following change of 
variables
\begin{equation}
  \begin{align}
  x_1 &= 1 + y_1/\sqrt{10} + y_2/5 + y_4/\sqrt{50}\\
  x_2 &= 1 + y_1/\sqrt{10} - y_2/5 + y_4/\sqrt{50}\\
  x_3 &= 1 + y_1/\sqrt{10} + y_3/5 - y_4/\sqrt{50}\\
  x_4 &= 1 + y_1/\sqrt{10} - y_3/5 - y_4/\sqrt{50}
  \end{align}
\end{equation}

In these coordinates, the quintic becomes

\begin{equation}
  5(\psi -1 ) =  y_1^2 + y_2^2 + y_3^2 + y_4^2 + O( \psi -1 )
\end{equation}


\todoin{Something about the branch}


\todoin{Volume of cycles wrapping singularities}
\todoin{Three-form integration}


\begin{equation}
  \Omega = \frac{x_5 dx_1 \wedge dx_2\wedge dx_3}{\pdv{p}{x_4}}
\end{equation}


\begin{equation}
  \int_{A^2} \Omega = \int  \dots
\end{equation}








