\chapter{Type IIA on the quintic}

Type IIA string theory on Calabi-Yau orientifolds with D6-branes wrapping special-Lagrangian three cycles provides
a framework to obtain semi-realistic particle physics models.
In order to understand better the physics of this construction, we choose a well-known compactification:
Fermat's quintic three-fold and try to expand the study to generic deformations.

\section{SLags on Fermat's quintic}
In this section, we describe Fermat's quintic and study the properties of SLag three-cycles 
defined on this manifold.

\subsection{Fermat's quintic}
The complex projective space $\mathbb{CP}^n$ is defined considering the complex space minus the origin $\mathbb{C}^{n+1}\backslash\qty{0}$
and establishing the equivalence relation $( z_0,\ldots, z_{n})\sim(\lambda z_0,\ldots,\lambda z_{n})$, $\lambda\in\mathbb C$.
%To emphasize that $z_i$ are homogeneous coordinates, they are sometimes denoted as $\qty[z_0:\ldots:z_{n}]$.
The local (non-homogeneous) coordinates $\xi^i$ of a $j$-patch where $z_j\neq0$ are obtained  in
terms of the homogeneous coordinates $z_i$ by choosing $\lambda=1/z_j$ so
\begin{equation}
  (\xi^1_j,\ldots,\xi^n_j)=\qty(\frac{z_0}{z_j},\ldots,\frac{z_{j-1}}{z_j},\frac{z_{j+1}}{z_j},\ldots,\frac{z_{n}}{z_j}).
\end{equation}

The complex projective space allows us to obtain lower dimensional Calabi-Yau manifolds as subspaces of $\mathbb{CP}^n$.
Indeed, an hypersurface of the projective space $\mathbb{CP}^n$ described by a (single) homogeneous polynomial of degree $d$ 
is a Calabi-Yau manifold (with vanishing first Chern class) if $d = n+1$. 
%Indeed, Chow's theorem states that any submanifold of complex dimension $n-r$ of $\mathbb{CP}^n$ can be realized as the zero locus of $r$-homogeneous polynomial equations.
An homogeneous polynomial $P$ of degree $d$ satisfies
\begin{equation}
  P(\lambda z_1,\ldots,\lambda z_n)=\lambda^d P(z_1,\ldots,z_n).
\end{equation}
We are interested in calculating the number of inequivalent $(n-1)$-dimensional submanifolds of $\mathbb{CP}^n$ that can be defined
through a polynomial equation of degree $d$.
The number of independent monomials of degree $d$ in $n+1$ variables is given by the binomial coefficient
\begin{equation}
  \# \text{ indep. monomials}={{d+(n+1)-1}\choose{(n+1)-1}}
\end{equation}
Not all of these lead to different manifolds, since some of them can be related through coordinate transformations, 
which belong to complex general linear group $GL(\mathbb C)_{n+1}$.
Thus, the number of possible submanifolds is given by
\begin{equation}
  \# \text{ submanifolds } \mathbb{CP}^n = \# \text{ indep. monomials}-\# \text{ components }GL(\mathbb C)_{n+1}
\end{equation}
Among of the ${{9}\choose{4}}-5^5=101$ possible submanifolds of $\mathbb{CP}^4$ defined by a quintic polynomial, lies
Fermat's quintic threefold defined by the polynomial $P_5$ as
\begin{equation}
  %P_4(z)& = z_0^5+z_1^5+z_2^5+z_3^5=0\\
  P_5(z) = z_0^5+z_1^5+z_2^5+z_3^5+z_4^5=0.
\end{equation}


\subsection{Construction of SLags}
When considering Type IIA string theory on a Calabi-Yau manifold, we mod out the orientifold 
projection (composed of a worldsheet parity and an orientifold involution) to obtain $\mathcal N=1$ supersymmetry.
A special-Lagrangian (SLag) three-cycle is constructed with the aid of the anti-holomorphic involution.
In the case of $\mathbb{CP}^4$, there is only one consistent anti-holomorphic involution \cite{Partouche2001},
which acts on the homogeneous coordinates as $\mathcal R: z_i \to\bar z_i$.
A particular instance of SLag three-cycle is defined as the fixed loci under $\mathcal R$.
In the case of Fermat's quintic, finding the fixed loci under $\mathcal R$ leads to the following SLag
\begin{equation}
  \qty{\qty[x_0:x_1:x_2:x_3:x_4]\in  \mathbb{RP}^4| x_0^5+x_1^5+x_2^5+x_3^5+x_4^5=0}.
\end{equation}
This subspace is topologically equivalent to $\mathbb{RP}^3$, which can be noticed through the
homeomorphism from $\mathbb{RP}^3$ to the SLag \cite{McLean1996}:
\begin{equation}
(u_0,u_1,u_2,u_3) \to (u_0,u_1,u_2,u_3, -(u_0^5+u_1^5+u_2^5+u_3^5)^{1/5}).
\end{equation}

In order to prove that the obtained subspace is indeed special-Lagrangian we must
verify that the Kähler form vanishes at the SLag and that it is calibrated with respect to the
same three-form as the $\Omega_3$ of the quintic \cite{Becker1995}.
\begin{itemize}
  \item The Kähler form $k$ transforms under the anti-holomorphic involution as $k\to -k$, 
    but since the pull-back of $k$ onto the three-dimensional subspace must be real and thus invariant under the anti-holomorphic involution, 
    the only possibility is $k\rvert_{\mathrm{SLag}}=0$

  \item 
The holomorphic three-form in a non-homogeneous patch where $x_0\neq0$ is
\begin{equation}
  \Omega_3=\frac{4}{2\pi i}\int \frac{x_0 dx_1\wedge dx_2 \wedge dx_3 \wedge dx_4}{P_5(x_i)}.
  \label{eq:3form}
\end{equation}
To show that the SLag is calibrated with respect to the Calabi-Yau three form,
we interpret $x_4$ as a function of $P_5$ and integrate over a loop around $P_5=0$
\begin{equation}
  \Omega_3=\frac{x_0 dx_1\wedge dx_2 \wedge dx_3}{x_4^4}
  =\frac{dy_1\wedge dy_2\wedge dy_3}{y_4^4}
  \label{eq:hol3}
\end{equation}
where we have defined the local coordinates $y_i=x_i/x_0$.
By taking the norm of the previous equation, we relate $\Omega_3$ to the determinant of the 
six-dimensional metric
\begin{equation}
  ||\Omega_3||^2 = \frac{1}{\det g |y_4|^8}.
\end{equation}
Since $\Omega_3$ is covariantly constant, $||\Omega_3||$ must also be proportional to a constant defined as
\begin{equation}
  ||\Omega_3||^2 = \alpha. %8e^{2\kappa}.
\end{equation}
The pull-back of the six-dimensional metric onto the three-cycle is
\begin{equation}
  h_{\alpha\beta}=2\partial_\alpha y^i \partial_\beta y^{\bar j}g_{i\bar j}.
\end{equation}
The volume form of the SLag three-cycle written in terms of the coordinates $\sigma^i$
defined on the three-cycle is
\begin{equation}
   \sqrt{\det h_{ab}}d\sigma^1\wedge d\sigma^2\wedge d\sigma^3
  = 8\alpha|\det \partial y|d\sigma^1\wedge d\sigma^2\wedge d\sigma^3
\end{equation}
which is proportional to \eqref{eq:hol3} and thus the SLag is calibrated with respect to $\Omega_3$.
\end{itemize}

It is possible to define different SLags by exploiting the $\mathbb Z_5^4$ symmetry of the quintic,
which leads to a whole family of SLags
\begin{equation}
  \ket{0,k_1,k_2,k_3,k_4} = {\qty[x_0:\omega^{k_1} x_1:\omega^{k_2} x_2:\omega^{k_3} x_3:\omega^{k_4} x_4]}
\end{equation}
where $\omega=e^{i\frac{2\pi}{5}}$. 
The freedom to choose $k_i\in \mathbb Z_5$ leads to $5^4=625$ SLag three-cycles, but not all of them are calibrated
with respect to the same three-form as O$6$-plane $\ket{0,0,0,0,0}$.
%
Concretely, only the SLags that satisfy $\sum_{i} k_i^4=0 \text{ mod 5}$ are calibrated with respect to the same three-form as the O$6$-plane.
Accounting for this constraint, there are $125$ SLags calibrated with respect to $\Omega_3$.

\subsection{Moduli space of SLags}

When we do model building with D-branes, we want to make sure that the gauge groups on the D-branes are not spontaneously broken by one D-brane in a stack being displaced from the rest.
In other words, D-branes that wrap cycles which can be deformed or displaced, come with additional open string moduli in the adjoint representation of the gauge group.
Geometrically, these open string moduli are played by the deformation moduli of cycles.
Roughly speaking, we can thus distinguish between two types of cycles:
rigid cycles, which do no carry deformation moduli
and non-rigid cycles, which carry deformation moduli.
The first class is perfect for model building, while the latter case requires more work to argue why the vev of the open string moduli is set to zero.

%We are not only interested in SLags themselves, but also in their moduli space.
%The reason is that deformations of SLag three-cycles codify the information of the open string 
%excitations of the D$6$-branes that wrap around them.
%In particular, they determine de vev of the scalar field excitations and thus the position of the D6-branes
%and whether they are rigid configurations or not.

A deformation of a reference SLag three-cycle $\Pi_3$ into another three-cycle (in the same homology class as the original SLag)  can be parameterized by a normal vector field, which belongs to the normal bundle of $\Pi_3$.  
This deformation normal vector field can be written in a basis $s^i$ as
\begin{equation}
  X=\phi_i s^i
\end{equation}
where $\phi_i$ correspond to the scalar fields defined on the D6-brane worldvolume theory.
The Kähler form introduces an isomorphism between the normal bundle of $\Pi_3$ and the cotangent bundle of $\Pi_3$.
According to McLean's theorem, harmonic one-forms are associated to a deformation vector field.
Recalling that there is a single harmonic form in each cohomology class, we associate to every $s^i$ a one-form $\xi_i\in H^1(\Pi)$.
As a consequence, the moduli space of SLags is identified with $H^1(\Pi_3)$.

A gauge potential $A$ on the D$6$-brane can be expanded in a basis of one-forms $\xi_i$ as $A=a^i \xi_i$.
In order to fill in the multiplets of the spectrum arising from open string excitation of D6-branes,
we must arrange the $a^i$ and $\phi_i$ into a complex scalar field
\begin{equation}
 \Phi^i = \phi_i + ia^i. 
\end{equation}

We can now apply the previous reasoning to SLag three-cycles defined on Fermat's quintic.
Since $H^1(\mathbb{RP}^3)\simeq H_1(\mathbb{RP}^3)\simeq\mathbb Z_2$, there are no continuous deformations of the
SLags and the SLags are guaranteed to remain rigid.

\subsection{Intersection numbers of SLags}
We now study the intersection numbers between different SLags, which are used to determine the number 
of chiral fermions associated to the intersecting D6-branes.
The computation was done in \cite{Brunner2000}.

First of all, we choose a coordinate patch of $\mathbb{CP}^4$ where $x_0=1$ and take one of the SLags to be $\ket{0,0,0,0,0}$
and the other SLag $\ket{0,k_1,k_2,k_3,k_4}$.
The latter SLag can  be  obtained by applying successive rotations $\omega_k$ associated to the $\mathbb Z_5^4$ symmetry
as $\ket{0,k_1,k_2,k_3,k_4}=\omega^{k_1}_1\omega^{k_2}_2\omega^{k_3}_3\omega^{k_4}_4\ket{0,0,0,0,0}$.
The intersection numbers in this patch are:
\begin{itemize}
  \item If $\omega_i^{k_i}\neq 1$ for all $i$, there are no intersections in this patch. 
    The justification is that the coordinates of one of the SLags $\qty[1:x_1:x_2:x_3:x_4]$ where $x_i\in\mathbb R$ take only real values,
    while some of the coordinates of the SLag $[1:\omega^{k_1}x_1:\omega^{k_2}x_2:\omega^{k_3}x_3:\omega^{k_4}x_4]$ take
    always complex numbers.

  \item If $\omega_i^{k_i}= 1$ for only one $i$, which we set to be $i=4$, the intersection 
    is at a single point.
    In particular, the intersection of $\qty[1:x_1:x_2:x_3:x_4]$ with $[1:\omega^{k_1}x_1:\omega^{k_2}x_2:\omega^{k_3}x_3:\omega^{k_4}x_4]$
    is located at $(1,0,0,0,x_4)$. 
    When we restrict this line in $\mathbb{CP}^4$ to the quintic hypersurface, the intersection reduces to the point $(1,0,0,0,-1)$.
    The signature of the intersection is $\mathrm{sgn}(\Im \omega^{k_1}\Im\omega^{k_2}\Im\omega^{k_3})$. 

  \item If $\omega_i^{k_i}=1$ for at least two different values of $i$, we have to deform one of the SLags. 
    The deformation must be normal to both SLags and through McLean's theorem  we identify the normal bundle of the intersection
    with its tangent space.
    Then, the intersection number is the number of zeros of the non-trivial vector fields 
    that can be defined on the intersection locus.
    \begin{itemize}
  \item
    If $\omega_i^{k_i}= 1$ for two $i$'s, the intersection is $\qty[1:0:0:x_3:x_4]$, which is homeomorphic to $\mathbb{RP}^1$ and the circle $S^1$.
    We can define a nowhere vanishing vector field on the circle, embedded in $\mathbb R^2$, associating to 
    every point  $(\sin\phi,\cos\phi)$ the vector $X(\phi)=-\sin\phi\partial_x +\cos\phi\partial_y$.
    Thus, the intersection number is zero $I_{\mathbb{RP}^1}=0$.
  \item 
    If $\omega_i^{k_i}= 1$ for three $i$'s, the intersection is $[1:0:0:x_3:x_4]$, which is homeomorphic to $\mathbb{RP}^2$.
    The double cover of $\mathbb{RP}^2$ is the two-sphere $S^2$, which we consider embedded in $\mathbb R^3$.
    We define the vector fields
    \begin{align}
      X_\theta& =\cos\phi\cos\theta \partial_x+\cos\theta\sin\phi\partial_y-\sin\theta\partial_z\\
      X_\phi&=\sin\theta(-\sin\phi\partial_x+\cos\phi\partial_y)
    \end{align}
    where $X_\phi$ vanishes at the poles, while $X_\theta$ is not (uniquely) defined at those points.
    These two zeros reduce to a single zero when we relate antipodal points of $S^2$ to a single point
    of $\mathbb{RP}^2$ by the $\mathbb Z_2$ action, so the intersection number is $I_{\mathbb{RP}^2}=1$.
    The orientation of the intersection is $\mathrm{sgn}(\Im \omega^{k_j})$ $j\neq i$.
  \item 
    If $\omega^{k_i}=1$ for all $i$'s, the intersection is $\mathbb{RP}^3$, whose double cover is $S^3$.
    Since the vector fields on $S^3$ have no zeros, there are no intersections $I_{\mathbb{RP}^3}=0$.
    \end{itemize}
\end{itemize}

The total intersection number is obtained by repeating this procedure in all patches
and then adding all the obtained intersection numbers.
It can be shown that the total intersection number of intersections of
$\ket{0,0,0,0,0}$ with $\ket{0,k_1,k_2,k_3,k_4}$ that preserve supersymmetry ($\sum_i k_i=1$) is zero. 
%the coefficient of the monomial $g_1^{k_1}g_2^{k_2}g_3^{k_3}g_4^{k_4}$ in
%\begin{equation}
%  \prod_{i=0}^4 (g_i+g_i^2-g_i^3-g_i^4)
%\end{equation}

\subsection{Volume of SLags}
\label{ss:vol}
We now try to compute the volume of SLag three-cycles, as they carry the information of the four-dimensional
gauge theory supported by the wrapping D$6$-branes, such as the gauge coupling.
It suffices to determine the volume of $\ket{0,0,0,0,0}$, since a generic SLag can be obtained through $\mathbb Z_5$ rotations.
The holomorphic three-form on a patch $x_0\neq0$ in terms of the local coordinates $y_i$ is
\begin{equation}
  \Omega_3=\frac{dy_1\wedge dy_2\wedge dy_3}{5y_4^4}
\end{equation}
The volume of the SLag is obtained integrating $\Omega_3$ over the three-cycle
\begin{equation}
  \mathrm{Vol}=\int_{\mathbb{RP}^3}\frac{dy_1\wedge dy_2\wedge dy_3}{5 y_4}=
\int \frac{dy_1\wedge dy_2\wedge dy_3}{5(1+y_1^4+y_2^4+y_3^4)^{4/5}}
\label{eq:int}
\end{equation}
It is not evident how to compute this integral, due to the non-trivial integration domain.

We can relate the volume of $\ket{0,0,0,0,0}$ to the volume of an associated three-sphere as follows.
The SLag three-cycle is homeomorphic to $\mathbb{RP}^3$, which in turn is diffeomorphic to the rotation group $SO(3)$. 
Considering that there is a map from $S^3$ with antipodal points identified onto $SO(3)$,
we conclude that the volume of a SLag is half the volume of a corresponding three-sphere on the quintic.
In practice, we can make no further progress through this approach, since we do not know
the Calabi-Yau metric of the quintic.

We come back to the integral \eqref{eq:int},
where the integration domain is given by the points $x_1,x_2,x_3 \in \mathbb R$ where $x_4=-(1+x_1^5+x_2^5+x_3^5)^{4/5}$ is real
forms the constraint that traces out the integration domain in the $\mathbb{R}^3$ spanned by ($x_i=1,2,3$).
Given the complexity of integrating over this hypersurface, we restrict the integration over the positive octant $0<x_i<\infty$, 
where the reality condition of $x_4$ is guaranteed.
Although this would only yield a lower bound of the volume, it can be used as an estimate of
the behaviour of the volume of SLags when we turn on the deformations of the quintic.
Over the positive octant, the integration can be carried out analytically and it yields as lower
bound of the total volume
\begin{equation}
  \mathrm{Vol}\geq\frac{\Gamma\qty(\frac{1}{5})\Gamma\qty(\frac{3}{10})\Gamma\qty(\frac{11}{10})\Gamma\qty(\frac{6}{5})}{5 \times 2^{1/5}\pi}\approx 0.61.
  \label{eq:vol}
\end{equation}

\subsection{Implications for model building}
Unfortunately, we have seen that the supersymmetric SLag three-cycles have all vanishing 
intersection numbers, which limits the applicability of intersecting D$6$-branes in model building.
The reason is that the number of chiral fermions is determined by the intersection numbers 
of the SLags on which D$6$-branes wrap.
A workaround to this problem consists in considering non-supersymmetric configurations involving three-cycles
which are not calibrated with respect to the same three-form as $\ket{0,0,0,0,0}$.
This allows to obtain a spectrum containing three generations of chiral fermions.
To cancel RR charges, a hidden sector of D-branes is introduced which does not intersect the 
Standard Model branes, which is the strategy followed in \cite{Blumenhagen1}\cite{Blumenhagen2}.
In the case when we have a supersymmetric configuration and no hidden spectrum, it suffices to introduce a stack of 4 coincident
D6-branes wrapping the SLag $\ket{0,0,0,0,0}$. Although this last configuration has no RR tadpoles, 
it leads to a $SO(8)$ gauge theory on the D6-branes, so it does not contain a chiral spectrum.

\section{SLags on the deformed quintic}

\subsection{Deformations of the quintic}

A generalization of Fermat's quintic consists in considering hypersurfaces defined by a generic polynomial of degree five
\begin{equation}
  P_5(z)=\sum_{n_0+n_1+n_2+n_3+n_4=5} a_{n_0 n_1 n_2 n_3 n_4} z_0^{n_0}z_1^{n_1}z_2^{n_2}z_3^{n_3}z_4^{n_4}=0
\end{equation}
This construction reduces to Fermat's quintic when $a_{50000}=a_{05000}=a_{00500}=a_{00050}=a_{00005}=1$ and all other coefficients are zero.
A well-know example of a deformation is
\begin{equation}
  z_0^5+z_1^5+z_2^5+z_3^5+z_4^5- 5\psi z_0z_1z_2z_3z_4=0
\end{equation}
 where the parameter $\psi$ can take three values of particular relevance
\begin{itemize}
  \item $\psi=0$: the Fermat point(or Gepner point).
  \item $\psi=1$: the conifold point.
  \item $\psi=\infty$: the large complex structure limit.
\end{itemize}

We can deform Fermat's quintic by adding a monomial to the defining polynomial.
The possible monomials are of the type:
\begin{enumerate}
  \item $z_0z_1z_2z_3z_4$, $1$ deformation.
  \item $z_i z_j z_k(z_l)^2$, ${5}\choose{3}$${2}\choose{1}$$=20$ deformations. 
  \item $z_i(z_j)^k(z_l)^2$, ${5}\choose{1}$${4}\choose{2}$$=30$ deformations.
  \item $z_i z_j(z_k)^3$, ${5}\choose{2}$${3}\choose{1}$$=30$ deformations.
  \item $(z_i)^2(z_j)^3$, ${5}\choose{1}$${4}\choose{1}$$=20$ deformations.
  \item $z_i(z_j)^4$, ${5}\choose{3}$${2}\choose{1}$$=20$ deformations.
  \item $(z_i)^5$, $5$ deformations.
\end{enumerate}
In total there are $126$ deformations, but they are not all independent, since they 
can be related through coordinate transformations.
Subtracting the $25$ components of $GL(5,\mathbb C)$ we obtain $101$ deformation parameters.
This is precisely the Hodge number $h_{2,1}=101$, since the defining homogeneous polynomial  
represents a particular choice of the complex structure, which is given by a
harmonic $(2,1)$-form in $H^{2,1}(X,\mathbb C)$.
The variations of the Kähler structure are given by the $h^{1,1}$ and
for the quintic there exists precisely one Ricci-flat Kähler form.
%there is precisely one possible deformation.

Fermat's quintic enjoys the $\mathbb Z_5^4$ symmetry acting on the homogeneous coordinates 
\begin{equation}
  z_i \to \omega^{k_i} z, \qquad \omega_i=e^{i\frac{2\pi}{5}}
  \label{eq:sim}
\end{equation}
which is broken into a smaller subgroup when turning on deformations.
We should also examine whether deformations introduce any singularities, which
are defined as the points 
satisfying the following equations simultaneously:
\begin{equation}
  P(X)=0, \qquad dP=0.
\end{equation}
The nature of the singularity is obtained by evaluating the Hessian at the singularity.
That is, calculating $d^2 P$ in a local coordinate patch, where one of the homogeneous coordinate is non-zero.
We proceed to summarize the singularities and associated symmetry subgroups of the deformations.
\begin{enumerate}
  \item No deformations.

    \begin{equation}
    z_0^5+z_1^5+z_2^5+z_3^5+z_4^5=0
    \end{equation}

    The only point where $dP_5=0$ is $(0,0,0,0,0)$, which 
    is not part of $\mathbb{CP}^4$, so Fermat's quintic is a smooth Calabi-Yau manifold. 
    It enjoys the $\mathbb{Z}_5^4$ symmetry mentioned in equation \eqref{eq:sim}.

  \item $z_0z_1z_2z_3z_4$

    \begin{equation}
    z_0^5+z_1^5+z_2^5+z_3^5+z_4^5- 5\psi z_0z_1z_2z_3z_4=0
    \end{equation}
    The singularities are located at the points:
    \begin{equation}
      z_i^5=\psi z_0z_1z_2z_3z_4.
    \end{equation}
    Multiplying them together
    \begin{equation}
      \prod_{i=0}^4 z_i^5=\psi^5 ( z_0z_1z_2z_3z_4 )^5.
    \end{equation}
    So in order to obtain a singular point, $\psi^5=1$.
    Taking $\psi=1$, there is a single singular point $(1,1,1,1,1)$ which is a node, since the Hessian doesn't vanish.
    Locally, the node can be recast into a conifold singularity.

    The associated symmetry group is reduced from $\mathbb{Z}_5^4$ to $\mathbb Z_5^3$.

  \item  If we apply the same procedure for the rest of deformations, where the defining polynomial is
    \begin{equation}
      z_0^5+z_1^5+z_2^5+z_3^5+z_4^5 - 5\psi(\text{deformation monomial})=0,
    \end{equation}
    we find that the symmetry group is $\mathbb Z_5^3$ and
    that for particular values of $\psi$ they develop
    ordinary double singularities on hypersurfaces of the quintic.
    This implies that in the presence of certain deformations we encounter singular surfaces or singular curves, whose mathematical features first have to be understood before these backgrounds can be applied in string compactifications.  
%    Since the Hessian vanishes at those singularities, an additional treatment would be needed to determine the nature of the singularity.
%  \item $z_0z_1z_2(z_3)^2$
%
%    $z_0^5+z_1^5+z_2^5+z_3^5+z_4^5- 5\psi z_0z_1z_2(z_3)^2=0$
%
%    Singularities in the coordinate patch where $z_3=1$
%    \begin{equation}
%      (\frac{1}{\sqrt[5]{2}}, \frac{1}{\sqrt[5]{2}}, \frac{1}{\sqrt[5]{2}}, 1,0) 
%    \end{equation}
%    $\psi=\frac{1}{\sqrt[5]{2}}$
%
%  \item $5\psi z_0(z_1)^2(z_2)^2$
%
%    $z_0^5+z_1^5+z_2^5+z_3^5+z_4^5- 5\psi z_0z_1z_2(z_3)^2=0$
%
%    \begin{equation}
%      (1, \sqrt[5]{2}, \sqrt[5]{2}, 0, 0)
%    \end{equation}
%    $\psi=\frac{1}{\sqrt[5]{2^4}}$
%
%  \item $5\psi z_0z_1(z_2)^3$
%
%    \begin{equation}
%      (\sqrt[5]{3}, \sqrt[5]{3},1, 0, 0)
%    \end{equation}
%    $\psi=\frac{1}{\sqrt[5]{3^3}}$
%
%  \item $5\psi (z_0)^2(z_1)^3$
%
%    \begin{equation}
%      (1, \sqrt[5]{\frac{3}{2}}, 0, 0, 0)
%    \end{equation}
%    $\psi=\frac{1}{\sqrt[5]{3^3}}$
%
%  \item $5\psi z_0(z_1)^4$
%
%    \begin{equation}
%      (1, \sqrt[5]{4}, 0, 0, 0)
%    \end{equation}
%    $\psi=\frac{1}{\sqrt[5]{4^4}}$
%
\end{enumerate}

\subsection{SLags on the deformed quintic}
Let us consider a specific deformation of Fermat's quintic
\begin{equation}
  z_0^5+z_1^5+z_2^5+z_3^5+z_4^5- 5\psi z_0z_1z_2z_3z_4=0.
\end{equation}
SLags are then defined as the fixed loci under the anti-holomorphic involution $\mathcal R: z_i \to \bar z_i$:
\begin{align}
  &\ket{0,0,0,0,0}_\psi= \nonumber \\
  &\quad\qty{\qty[x_0:x_1:x_2:x_3:x_4]\in  \mathbb{RP}^4| x_0^5+x_1^5+x_2^5+x_3^5+x_4^5-5\psi x_0x_1x_2x_3x_4=0}.
\end{align}

There is still a $\mathbb Z_5^3$ symmetry which allows us to define additional SLags by successive rotations:
\begin{align}
 &\ket{0,k_1,k_2,k_3,4k_1+4k_2+4k_3}_\psi=\nonumber\\
 &\qquad\qty{\qty[x_0:\omega^{k_1}x_1:\omega^{k_2}x_2:\omega^{k_3}x_3:\omega^{4k_1+4k_2+4k_3}x_4]\in  \mathbb{RP}^4|x_i\in\ket{0,0,0,0,0}_\psi}.
\end{align}
The possible combinations of $k_i$ lead to $125$ supersymmetric SLag three-cycles
calibrated with respect to the same holomorphic three-form as $\ket{0,0,0,0,0}_{\psi}$.
For this reason, there are the same number of supersymmetric SLags as on the undeformed Fermat's quintic.
It should be noted that the $500$ SLags calibrated which are not calibrated with respect to the same three-form as $\ket{0,0,0,0,0}$ 
do not appear at all.
 
To determine the topology of the SLags, we try to form an homeomorphism between $\mathbb{RP}^3$ and the SLags.
When the deformation is non-zero, it seems impossible to invert one of the variables of the defining
quintic equation in terms of the others.
Thus, we consider small values of $\psi$ to obtain an approximation through the Newton-Raphson method
and assume that we can analytically continue it for all values of $\psi$.
The homeomorphism from $\mathbb{RP}^3$ to the SLags on the deformed quintic would be to first order in $\psi$
\begin{align}
  &(u_0,u_1,u_2,u_3) \to \bigg(u_0,u_1,u_2,u_3,\nonumber\\
  &\quad-(u_0^5+u_1^5+u_2^5+u_3^5)^{1/5} \qty(1+\frac{\psi u_0u_1u_2u_3}{(u_0^5+u_1^5+u_2^5+u_3^5)^{4/5}-\psi u_0u_1u_2u_3}+O(\psi^2))\bigg).
\end{align}
So the topology of SLags is the same as in the  undeformed case and SLags are rigid.

The intersection numbers of SLags are not expected to change with respect to those obtained for Fermat's
quintic, since they are topological quantities.
Indeed, it can be shown that the intersection numbers are the same as those on Fermat's quintic, so
the same obstruction to model building applies.

We now compute the volume of a SLag for small values of $\psi<1$.
In order to solve the hypersurface equation in terms of an expansion of a small parameter, 
we define $\Delta$, $\eta$ and $u$ \cite{Candelas1991} as
\begin{equation}
\Delta = 1 + x_1^5 + x_2^5 +x_3^5, \qquad \eta = \Delta^{-1/5} x_4, \qquad u = \Delta^{-4/5} x_1 x_2 x_3,
\end{equation}
so the hypersurface equation becomes
\begin{equation}
  \Delta (\eta^5+1-5\psi u\eta)=0.
\end{equation}
Since $u$ is bounded as $0\leq u \leq 4^{-4/5}$, we can obtain a solution in powers of
$\psi u$ through the Newton-Rapshon method.
A particular solution to first order in $\psi$ is given by
\begin{equation}
  \eta=-1-\frac{u\psi}{1-u\psi}=-\frac{1}{1-u\psi}
\end{equation}
so 
\begin{equation}
  x_4=-\frac{\Delta^{1/5}}{1-u\psi}.
\end{equation}
Inserting this expression into the three-form, we obtain the volume for of
the SLag to second order in $\psi$ 
\begin{equation}
  \Omega_3=\frac{dx_1\wedge dx_2\wedge dx_3}{5\Delta^{4/5}}\qty(1-3u\psi-u^2\psi^2+O(\psi^3)).
\end{equation}
The volume of the SLag would be obtained integrating this three-form over the hypersurface that depends on $\psi$.
Fortunately, the integration domain is the same as when $\psi=0$.
Still, we cannot compute analytically this integral, so we restrict the integration region to
the positive octant, as we did before in section ~\ref{ss:vol} ($0<x_i<\infty$).
The integral of each term of the $\psi$ expansion can be obtained analytically in terms of gamma functions
and it yields the lower bound of the volume of SLags
\begin{align}
  \mathrm{Vol}\geq&
\frac{\Gamma\left(\frac{1}{5} \right) \Gamma\left(\frac{3}{10}\right)  \Gamma\left(\frac{11}{10}\right)  (\Gamma\left(\frac{6}{5} \right))^2}{5 \times 2^{1/5} \pi} 
  -  \psi  \frac{3 \pi \Gamma\left( \frac{2}{5}\right) (\Gamma\left( \frac{7}{5}\right))^2}{100 \times 2^{2/5} \Gamma\left( \frac{9}{10} \right) \Gamma(\frac{13}{10})} \\
  &-  \psi^2 \frac{\pi (\Gamma\left( \frac{3}{5}\right))^3}{125 \times 2^{3/5} \Gamma\left( \frac{1}{10} \right) \Gamma\left( \frac{17}{10} \right)} + O(\psi^3)\\
 \approx& 0.61-0.13\psi-0.0063\psi^2+O(\psi^3)
\end{align}
In order to compare this volume with the volume of the SLag at the Fermat point, we divide
the result by the volume obtained in equation \eqref{eq:vol}
\begin{equation}
  \mathrm{Vol}/\mathrm{Vol}|_{\psi=0} \approx 1 - 0.21\psi - 0.01\psi^2 + O(\psi^3)
\end{equation}
and observe a linear decrease in the lower bound of the volume.
This implies that the gauge theory living on a D6-brane wrapped on such a SLag three-cycle will have an increasing gauge coupling away from the Fermat point.
