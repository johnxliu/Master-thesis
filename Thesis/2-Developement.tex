\chapter{D6-branes on Calabi-Yau manifolds}

\subsection{Types of string theories}


\subsection{Compactification}

In the following section we motivate the requirement that additional dimensions are compactified
over a Calabi-Yau manifold (a compact complex manifold of $SU(3)$ holonomy).

We assume that the manifold $\mathcal M$ is factorizable into a four-dimensional maximally symmetric space-time $N$ and a six-dimensional compact space $K$,
$\mathcal M =  N\times K$.

Out of all the possible ways to compactify a theory, we pick out the ones that preserve some
degree of supersymmetry.
%There are several reasons for this choice:
%\begin{itemize}
%%  \item Gauge hierarchy problem.
%%    \todoin{Is it relevant today?}
%  \item As a way to solve the equations of motion.
%    \todoin{What does this mean? A SUSY configurations satisfies the eom -> SUGRA. Classically?}
%  \item It gives a nice phenomenological description.
%
%    A four dimensional theory with $\mathcal N=1$ supersymmetry allows for massless fermions
%    that transform in a complex representation of the gauge group associated to the supersymmetry.
%    Since $\mathcal N\geq 2$ in four dimensions all fermions must transform in a real representation 
%    of the gauge group, we shall only consider the case $\mathcal N=1$.
%
%    \todoin{Understand the reason for different possible representations.}
%    \todoin{Does this mean that at higher energies there are no chiral fermions?}
%\end{itemize}
Our main reason for this choice is that they provide a nice way to obtain realistic particle 
physics models. 
In particular, we will see that an $\mathcal N=1$ supersymmetric theory allows for chiral fermions.
In addition, supersymmetric configurations are easier to study before tackling more general compactifications.

The supersymmetry condition means there is a transformation the relates fermions to bosons, and
vice versa.

Superspace
Roughly, extend a Poincaré algebra to in supersymmetry generators which satisfy specific 
anti-commutation relations, instead of commutation relations
\begin{equation}
  \{ Q, Q\} = \{\bar Q,\bar Q} = 0, \{Q, \bar Q}=
\end{equation}

A finite SUSY transformation is
\begin{equation}
  e^{i(\eta Q+ \bar \eta \bar Q - x_\mu p^\mu)}
\end{equation}

Every supersymmetry transformation is parametrized by an infinitesimal parameter $\eta_\alpha (X)$,
so that 
\begin{equation}
  \delta \ket{boson} = \ket{fermion} 
\end{equation}
\todoin{Insert definition of $\eta$. Is it a Grassmann variable, a spinor, an operator\ldots}
%which is anti-commuting, two-component Weyl fermion that has an associated conserved supercharge $Q$ at every space-time point.

\todoin{Fill in the proof. We translate field equations into operator equations.}
\todoin{What is the relation of the charge to the full SUSY transformation.} 
A conserved charge $Q$ associated to an unbroken supersymmetry annihilates the vacuum $\ket{\Omega}$,
so $Q\ket{\Omega}=0$, since
This in turn means that for any operator $U$, $\ev{ \{Q,U\} }{\Omega}=0$.
\begin{equation}
  U' = U+\delta U = e^{-iQ\eta} U e^{iQ\eta} = (1 + iQ\eta) U (1- iQ\eta) = \cdots
\end{equation}
If $U$ is a fermionic operator, we derive that the variation of the operator under the supersymmetry
transformation is $\delta U = \{ Q, U\}$.
Taking this as the classical limit, $\delta U = \ev{\delta U}{\Omega}$.
Thus, we conclude that at tree level $\delta U = 0$ for any fermionic field $U$.
\todoin{What is really $U$?. Read about tree level.}

The low-energy theory of the ten-dimensional type IIA string theory is type IIA SUGRA, which although it has
$\mathcal N=2$ instead of $\mathcal N=1$, we will see a workaround. 
The type IIA string theory has as NS-NS fields the dilaton $\phi$, the metric $G$ and the two-form field $B$.
In the RR sector we have the one-form and three-form, $c_1$ and $c_3$.
For convenience, we define employ the democratic formulation
\todoin{Understand democratic formulation}
We also define the field strength of the NS form $H=dB$ and the $F^{10)}$ as the formal sum
\begin{equation}
  F^{10)}=dC-H\wedge C+me^B
\end{equation}
where $C$ is the formal sum of odd differential form fields.
\todoin{What is m?}
\begin{equation}
  F^{10)}_n = (-1)^{\floor{n/2}}\star F^{10)}_{10-n}
\end{equation}

Type IIA SUGRA has as elementary fermions two Majora-Weyl gravitinos of the same chirality $\psi_M$ and two dilatinos $\lambda$. 
Their variation under a supersymmetry transformation is
\begin{equation}
    \delta \psi_M = D_M \eta + \frac{1}{4}\slashed H_M \Gamma_{11} \eta + \frac{1}{16}e^\phi \sum_n \slashed F^{10)}_n \Gamma_M \Gamma_{11}^{n/2)} \sigma^1 \eta
\end{equation}

Where $H=dB$ is the field strength of the antisymmetric two-form $B$.
$\slashed{F}_n^{10)} = \frac{1}{n!}F_{P_1\ldots P_N}\Gamma^{P_1\ldots P_N}$ and
$\slashed H_M= \frac{1}{2} H_{MNP} \Gamma^{NP}$.
The $\Gamma$ $\sigma^1$
\todoin{
  Where the Dirac matrices for the ten-dimensional space time are
  $\Gamma^M = e^M_A \Gamma^A.$
   Here $e^M_A$ denotes the vielbein that describes the graviton and $\Gamma^A$ are elements of a 
Clifford algebra, so $\{\Gamma^A,\Gamma^B\}=2\eta^{AB}$.
}

Supersymmetry preservation means that all variations must be zero. 
For convenience, we set $H=0$ and $\phi=\mathrm{const.}$ .
This leads to the constraint that $\eta$ is a covariantly constant spinor
\begin{equation}
  \delta \psi_M &= D_M \eta = 0
%  \delta \xi^a &= -\frac{1}{4g\sqrt \phi} \Gamma^{MN} F^a_{MN} \eta 
\end{equation}

\todoin{Why this cov der.}
The equation implies that there exists $[D_M,D_N]\eta=R_{MNPQ} \Gamma^{PQ} \eta=0$.
If we particularize to $T$, which is a maximally symmetric space, the last equation imposes that
$T$ is Minkowski space, which is not surprising.
Cosmological constant blah, blah, blah.
We can now use the first equation to conclude that $\eta$ does not depend on the uncompactified 
coordinates,  $\partial_T \eta=0$.

\todoin{\url{https://groups.google.com/forum/#!topic/sci.physics.research/rrBoIXk9Rw0}}
We proceed to examine what the existence of a covariantly constant spinor field imposes on the compact space. 

Let us consider a Riemannian manifold $K$ of dimension $n$ with a spin connection $\omega$, which 
is in general a $SO(n)$ gauge field.
If we parallel transport a field $\psi$ around a contractible closed curve $\gamma$, the field becomes
$U\psi$ where $U=\mathcal P e^{\int_\gamma dx \omega}$, where $\mathcal P$ is the path-ordered product (complete).

The set of the transformation matrices associated to all possible loops form the holonomy group of the manifold, 
which must be a subgroup of $SO(n)$.

\todoin{Fill in Group Theory discussion}

We now consider how the $SU(3)$ holonomy translates into the manifold. 

\todoin{Fill in Group Theory discussion}

The only  $U(3)$ invariants in the $\mathbf{6}$ representation of $SO(6)$ are the identity and
$\bar I$.

\subsection{U(3) holonomy implies complex manifold}

\todoin{Check the different spaces we consider.}
We can also form a tensor field on $K$ of the type $J^i_j(y)=g^{ik}(y) \bar\eta \Lambda_{kj} \eta(y)$.
For each point $y$, we can consider $J^i_j$ as a matrix that acts on the tangent space, so $v^i \to J^i_j v^j$.
In this sense, $J^i_j$ is a real, traceless and $SU(3)$ invariant matrix, which means that it must be proportional
to $\bar I$.
We had already seen that $\bar I = -I$, this an example of an almost-complex structure, which is 
a tensor field $J$ that satisfies $J^2=-I$.

If we employ complex coordinates, we can diagonalize $J$ so that the non-zero components are
$J^a_b=i\delta^a_b$ and $J^{\bar a}_{\bar b}=-i\delta^{\bar a}_{\bar b}$. This particular choice
is know as the canonical form.



It is always possible to choose particular coordinates to bring $J$ to the canonical form at a particular 
point.
But in general, the canonical form will not hold at an open neighborhood of a point.
If a manifold admits a set of coordinates (called local holomorphic coordinates) such that at every
point, the canonical form holds for an open neighborhood, then the almost complex structure is integrable.

The necessary and sufficient condition for integrability is that the Nijenhuis tensor

\begin{equation}
  N^k_{ij}= J^l_i(\partial_l J^k_j - \partial_j J^k_l) - J_j^l (\partial_l J^k_i - \partial_i J^k_l)
\end{equation}

vanishes.
An integrable almost-complex structure is a complex structure and a manifold with a complex structure
is a complex manifold.

\todoin{Coordinate definition of complex manifold}

\subsection{SU(3) implies vanishing first Chern class}
\todoin{Probably not necessary}

\subsection{If we want chirality we need SU(3) exactly}


\subsection{Moduli space}


\todoin{Calabi-Yau definition: SU(n) global holonomy<->non-vanishing n-form -> vanishing first Chern class <->vanishing Ricci curvature<->local 
SU(n) holonomy. If simply connnected, both definitions are equivalent.}
We start from a complex manifold with a metric st. it is a Calabi-Yau manifold.
We then deform it's metric while it remains CY


Given a Calabi-Yau manifold with a particular complex structure and metric, we could ask ourselves
when the deformations


ORIENTIFOLD PLANES AND D-BRANES

\subsection{Orientifold planes and D-branes}

\todoin{D-brane motivation through O-planes.}
If we compactify a Type II string theory on a Calabi-Yau manifold, we  obtain a four-dimensional
$\mathcal N=2$ supersymmetric theory.
In order to allow for chirality, we must obtain a $N=1$ theory. This can be done through the orientifold
projection, which consists in modding out $\Omega \mathcal R$ acting on the 10d manifold, 
where $\Omega$ is the worldsheet parity and $\mathcal R$ is a particular $\mathbb Z_2$ symmetry.


\todoin{Understand RR charges.}
\begin{equation}
  S= -\frac{1}{4\pi \alpha'}\int d^2\sigma \epsilon^{\alpha\beta} B_{\mu\nu}  \partial_\alpha X^\mu \partial_\beta X^\nu
\end{equation}

\subsection{D6-branes in flat 10d space}

We have seen that if we compactify an heterotic string theory on a Calabi-Yau manifold, we obtain
$\mathcal N=1$ which allows chirality. 
This is not the case of Type II theories, which lead to $\mathcal N=2$.
In order to reduce the degree of supersymmetry and thus obtain chiral 4d fermions, two D6-branes in flat 10d can intersect over a 4d region.
Before considering D6-branes over a Calabi-Yau compactification, we consider the flat space
case.

The open string spectrum of an intersection of a stack of $N_1$ D6-branes and a stack of $N_2$ D6-branes
can be classified as:

\begin{itemize}
  \item Strings stretching from one stack to itself, which lead to 7d $U(N_i)$ gauge bosons, three real
    adjoint scalars and their fermion superpartners, .
  \item String that go from one stack to the other are localized at the intersection. 
    Their associated fields are are charged under the bi-fundamental representation $(N_1, \bar N_2)$ of 
    $U(N_1)\times U(N_2)$, which includes a 4d chiral fermion in $(N_1,\bar N_2)$.
\end{itemize}

\subsection{D6-branes on a torus}

Let us consider type IIA theory compactified on a 6-torus $T^6=T^2 \times T^2 \times T^2$.

\subsection{D6-branes on a Calabi-Yau}
In order to obtain stable D6-brane configurations, we impose that they wrap around volume 
minimizing 3-cycles, which are special Lagrangian 3-cycles and satisfy
\begin{equation}
  J|_\Pi = 0 , \qquad \Im (e^{-i\phi}\Omega_3)|_\Pi=0
\end{equation}

The volume of the special Lagrangian 3-cycle is
\begin{equation}
  Vol(\Pi)=\int_\Pi \Re(e^{-i\phi}\Omega_3)
\end{equation}

\todoin{Spectrum}



INTRODUCE NON-ABELIAN GAUGE BOSONS

\subsection{Model building}
