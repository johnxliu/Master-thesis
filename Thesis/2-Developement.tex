\chapter{D6-branes on Calabi-Yau manifolds}

TYPES OF STRING THEORIES



ROLE OF COMPACTIFICATION



TYPES OF COMPACTIFICATIONS (Green, Uranga)

%It is well known that any physical theory that describes our universe must allow for chiral 
%fermions.
%The are several ways to compactify a string theory in order to obtain chirality.

We assume that the manifold is factorizable into a four-dimensional space and the compact space
$\mathcal M =  N\times K$.

\todoin{TIIA or TIIB?}
\todoin{For what theories is the following reasoning exactly valid.}

\todoin{Reason for N=1, chirality or th other reasons. I think N=something for other reasons and
N=1 for chirality.
}

Out of the possible ways to compactify a theory, we pick out the ones that preserve some
degree of supersymmetry.
There are several reasons for this choice
\begin{itemize}
  \item Gauge hierarchy problem.
  \item As a way to solve the equations of motion.
  \item It gives a nice phenomenological description.

    A four dimensional theory with $\mathcal N=1$ supersymmetry allows for massless fermions
    that transform in a complex representation of the gauge group associated to the supersymmetry.
    Since $\mathcal N\geq 2$ in four dimensions all fermions must transform in a real representation 
    of the gauge group, we shall only consider the case $\mathcal N=1$.

    \todoin{Understand the reason for different possible representations.}
\end{itemize}

\todoin{Understand the meaning of SUSY parameter.}
Every supersymmetry transformation is parametrized by an infinitesimal parameter $\eta_\alpha (X)$
which has an associated conserved supercharge $Q$ at every space-time point.

\todoin{Fill in the proof. We translate field equations into operator equations.}
A conserved charge $Q$ associated to an unbroken supersymmetry annihilates the vacuum $\ket{\Omega}$,
so $Q\ket{\Omega}=0$, since
This in turn means that for any operator $U$, $\ev{ \{Q,U\} }{\Omega}=0$.
\begin{equation}
  U' = U+\delta U = e^{-iQ\eta} U e^{iQ\eta} = (1 + iQ\eta) U (1- iQ\eta) = \cdots
\end{equation}
If $U$ is a fermionic operator, we derive that the variation of the operator under the supersymmetry
transformation is $\delta U = \{ Q, U\}$.
Taking this as the classical limit, $\delta U = \ev{\delta U}{\Omega}$.
Thus, we conclude that at tree level $\delta U = 0$ for any fermionic field $U$.
\todoin{What is really $U$?. Read about tree level.}

\todoin{Read some details on how to obtain this. Fill in the gaps. Check the implication direction.}
The low energy spectrum of a ten-dimensional theory has as elementary fermions the gravitino $\psi_M$,
the dilatino $\lambda$ and the gluino $\xi$.
Their variation is
\begin{equation}
  \begin{align}
  \delta \psi_M &= \frac{1}{\kappa }D_M \eta + \frac{\kappa}{32g^2 \phi}(\Gamma_M^{NPQ} - 9\delta^N_M \Gamma^{PQ})\eta H_{NPQ} + (\mathrm{Fermi})^2 \\
  \delta \xi^a &= -\frac{1}{4g\sqrt \phi} \Gamma^{MN} F^a_{MN} \eta + (\mathrm{Fermi})^2 \\
  \delta \lambda &=- \frac{1}{\sqrt 2 \phi}(\Gamma \partial \phi)\eta + \frac{\kappa}{8\sqrt 2 g^2 \phi} \Gamma^{MNP} \eta H_{\mnp} + (\mathrm{Fermi})^2
  \end{align}
\end{equation}

Supersymmetry preservation means that all variations must be zero. 
For convenience, we set $H=0$ and $\phi=\mathrm{const.}$ .
This leads to the constraints
\begin{equation}
  \begin{align}
  \delta \psi_M &= \frac{1}{\kappa }D_M \eta\\ 
  \delta \xi^a &= -\frac{1}{4g\sqrt \phi} \Gamma^{MN} F^a_{MN} \eta 
  \end{align}
\end{equation}

The first equation implies that there exists $[D_M,D_N]\eta=0$ $R_{MNPQ} \Gamma^{PQ} \eta$.
If we particularize to $T$, which is a maximally symmetric space, the second equation imposes that
$T$ is Minkowski space, which is not surprising.
Cosmological constant blah, blah, blah.
We can now use the first equation to conclude that $\eta$ does not depend on the uncompactified 
coordinates,  $\partial_T \eta=0$.

\todoin{\url{https://groups.google.com/forum/#!topic/sci.physics.research/rrBoIXk9Rw0}}
We proceed to examine what the existence of a covariantly constant spinor field imposes on the compact space. 

Let us consider a Riemannian manifold $K$ of dimension $n$ with a spin connection $\omega$, which 
is in general a $SO(n)$ gauge field.
If we parallel transport a field $\psi$ around a contractible closed curve $\gamma$, the field becomes
$U\psi$ where $U=\mathcal P e^{\int_\gamma dx \omega}$, where $\mathcal P$ is the path-ordered product.

The set of the transformation matrices associated to all possible loops form the holonomy group of the manifold, 
which must be a subgroup of $SO(n)$.

\todoin{Fill in Group Theory discussion}

We now consider how the $SU(3)$ holonomy translates into the manifold. 

\todoin{Fill in Group Theory discussion}

The only  $U(3)$ invariants in the $\mathbf{6}$ representation of $SO(6)$ are the identity and
$\bar I$.

\todoin{Check the different spaces we consider.}
We can also form a tensor field on $K$ of the type $J^i_j(y)=g^{ik}(y) \bar\eta \Lambda_{kj} \eta(y)$.
For each point $y$, we can consider $J^i_j$ as a matrix that acts on the tangent space, so $v^i \to J^i_j v^j$.
In this sense, $J^i_j$ is a real, traceless and $SU(3)$ invariant matrix, which means that it must be proportional
to $\bar I$.
We had already seen that $\bar I = -I$, this an example of an almost-complex structure, which is 
a tensor field $J$ that satisfies $J^2=-I$.

If we employ complex coordinates, we can diagonalize $J$ so that the non-zero components are
$J^a_b=i\delta^a_b$ and $J^{\bar a}_{\bar b}=-i\delta^{\bar a}_{\bar b}$. This particular choice
is know as the canonical form.

\todoin{Complex structure stuff}

\todoin{Nijenhuis tensor}

\begin{equation}
  N^k_{ij}= J^l_i(\partial_l J^k_j - \partial_j J^k_l) - J_j^l (\partial_l J^k_i - \partial_i J^k_l)
\end{equation}

$N=0$

\todoin{Coordinate definition of complex manifold}




INTRODUCE NON-ABELIAN GAUGE BOSONS
