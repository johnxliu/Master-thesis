\chapter{Type IIA compactifications}
In the following section we motivate the requirement that additional dimensions are compactified
over a Calabi-Yau manifold.

As we have seen in the previous chapter, Type IIA superstring theory requires nine spatial dimensions and one time dimension for consistency, yet our universe only consists of a four-dimensional spacetime continuum. This implies that six spatial dimensions have to be compactified on an internal manifold with an unobservably small volume.
We assume that the manifold $M$ is factorizable into a four-dimensional maximally symmetric space-time $T$ and a six-dimensional compact space $K$,
$M =  T\times K$.

Type IIA string theory on 10 dimensional flat space has a large degree of supersymmetry,
but the compactification choice can either preserve some degree of supersymmetry in four dimensions or remove it completely.
We will consider compactifications over an internal manifold that leave some supersymmetry in four dimensions intact.
A historical motivation for this choice is that they provide a nice way to obtain realistic particle 
physics models. 
In particular, we will see that a $\mathcal N=1$ supersymmetric theory allows for chiral fermions in four dimensions, while field theories with a higher number of supersymmetry in four dimensions do not.
In addition, supersymmetric configurations are easier to study before tackling more general compactifications.
The main reason is that supersymmetric compactifications of string theory allow  for stable dimensional reductions, whose higher-dimensional corrections can be systematically studied.

The algebra of a $\mathcal N=1$ supersymmetric theory in four-dimensional Minkowski spacetime is an extension of the Poincaré algebra by adding
 supersymmetry generators which satisfy specific anti-commutation relations, instead of commutation relations. 

\todoin{Reformulate. We want to obtain a condition for a susy vacuum in 4d. We can obtain a classical field theory constraint.}
A conserved charge $Q$ associated to an unbroken supersymmetry annihilates the vacuum $\ket{\Omega}$,
so $Q\ket{\Omega}=0$.
This in turn means that for any operator $U$, $\ev{ \{Q,U\} }{\Omega}=0$.
If $U$ is a fermionic operator, we derive that the variation of the operator under the supersymmetry
transformation is $\delta U = \{ Q, U\}$.
Taking this as the classical limit, $\delta U = \ev{\delta U}{\Omega}$.
Thus, we conclude that at the classical level $\delta U =\ev{ \{Q,U\} }{\Omega}= 0$ for any fermionic field $U$.

We now consider the SUGRA theory of type IIA string theory and
the condition that some four-dimensional supersymmetry remains.
In the same way that a translation generated by the momentum operator is parametrized by a vector and
a rotation is a parametrized by an antisymmetric tensor, a supersymmetry transformation generated
by $Q_\alpha$ is parametrized 
by a spinor $\eta_\alpha$.
The variation of the gravitino field under a supersymmetry transformation is
\begin{equation}
  \delta \psi_M = D_M \eta + \mathrm{(fluxes)}.
\end{equation}

Where $D_M$ is the covariant derivative on $M$.
Supersymmetry preservation means that all variations must be zero. 
Assuming that all fluxes vanish, this leads to the constraint that $\eta$ is a covariantly constant ten-dimensional spinor
\begin{equation}
  \delta \psi_M = D_M \eta = 0.
%  \delta \xi^a &= -\frac{1}{4g\sqrt \phi} \Gamma^{MN} F^a_{MN} \eta.
  \label{eq:cov}
\end{equation}

%The equation implies that there exists a spinor $\eta$ such that  $[D_M,D_N]\eta=R_{MNPQ} \Gamma^{PQ} \eta=0$.
%If we particularize this equation to the four-dimensional space-time  $T$, which is a maximally symmetric space, 
%it imposes that $T$ is Minkowski space and thus, $\eta$ only depends on the compact coordinates.
To study the implication of this equation to the four-dimensional space-time $T$, we employ the 
fact that $T$ is maximally symmetric, so we can decompose  the metric as
\begin{equation}
  ds^2=e^{2A(y)}\tilde {g}_{\mu\nu} dx^\mu dx^\nu + g_{mn} dy^m dy^n, \qquad  \mu=0,1,2,3 \quad m=1,\dots,6
\end{equation}
where $x^\mu$ are the compact coordinates, $y^m$ the internal coordinates and $\tilde{g}_{\mu\nu}$ can 
be either the de Sitter, anti-de Sitter or the Minkowski metric in four dimensions.
%\todoin{\url{https://groups.google.com/forum/#!topic/sci.physics.research/rrBoIXk9Rw0}}

Particularizing to the space-time components, equation \eqref{eq:cov} can be written as
\begin{equation}
  \widetilde{\nabla}_\mu \eta + \frac{1}{2}\qty(\tilde{\gamma}_\mu \gamma_5 \otimes \slashed \nabla A) \eta=0
\end{equation}
where $\widetilde{\nabla}$ and $\tilde \gamma_\mu$ are the covariant derivative and gamma matrix with respect $\tilde g_{\mu\nu}$.
This equation leads to the integrability condition
\begin{equation}
  \qty[\widetilde \nabla_\mu,\widetilde \nabla_\nu]\eta=\frac{1}{2}\qty(\nabla_m A)\qty(\nabla^m A)\gamma_{\mu\nu}\eta.
  \label{eq:1}
\end{equation}

On the other hand, the definition of the Riemann tensor is
\begin{equation}
  \qty[\widetilde \nabla_\mu,\widetilde \nabla_\nu]\eta=\frac{1}{4}\tilde R_{\mu\nu\lambda\rho} \gamma^{\lambda\rho}\eta.
  \label{eq:2}
\end{equation}
In the case of a maximally symmetric space, the Riemann tensor is $R_{\mu\nu\lambda\rho}=k(g_{\mu\lambda}g_{\nu\rho}-g_{\mu\rho}g_{\nu\lambda})$, where
$k$ is negative for anti-de Sitter, zero for Minkowski and positive for de Sitter.
Combining equations  \eqref{eq:1} and \eqref{eq:2}, and inverting $\gamma^{\mu\nu}$, we obtain 
\begin{equation}
  k + \nabla_m A \nabla^m A =0.
\end{equation}
Owing to the fact that on a compact manifold the only constant value of $(\nabla A)^2$ is zero,
we conclude that $k=0$ and thus the four-dimensional space-time must be Minkowski space.

ten-dimensional covariantly constant spinor $\eta$ into $\xi$ $\chi$


The existence of a covariantly constant spinor implies  for a type IIA theory that there are two four-dimensional supersymmetry parameters and therefore, $\mathcal N =2$.
\todoin{Explain better.}

\section{Type IIA on Calabi-Yau manifolds}
We examine more closely what the existence of a covariantly constant spinor field implies on the compact space. 


Let us consider a Riemannian manifold $K$ of dimension six with a spin connection $\omega$, which 
is in general a $SO(6)$ gauge field.
If we parallel transport a field $\psi$ around a contractible closed curve $\gamma$, the field becomes
$\psi'=U\psi$ where $U=\mathcal P e^{\int_\gamma dx \omega}$ and $\mathcal P$ denotes the path ordering of 
the exponential.
The set of transformation matrices associated to all possible loops form the holonomy group of the manifold, 
which must be a subgroup of $SO(6)$.

A covariantly constant spinor is left unchanged when parallel transported along a contractible
closed curve, so the holonomy matrices of a manifold that admits a covariantly constant spinor 
must satisfy $U\eta = \eta$.
Taking into account the Lie algebra isomorphism $\mathfrak{so}(6)\simeq \mathfrak{su}(4)$ we identify the positive
(negative)-chirality spinors of $SO(6)$ with the fundamental $\mathbf 4$ ($\mathbf {\bar 4}$)
of $SU(4)$.
Let us consider that $\eta$ is a positive chirality spinor, so it transforms according with the 
$\mathbf 4$ of $SU(4)$.
In order to have a covariantly constant spinor, the holonomy group must be such that the $\mathbf 4$
representation decomposes into a singlet.
This decomposition is achieved if the holonomy group is $SU(3)$ so that
 \begin{align}
  SO(6)  &\to SU(3)\\
  \mathbf 4 &\to \mathbf 3 \oplus \mathbf 1
\end{align} 
%Then, $\eta$ is left invariant under $SU(3)$ transformations and can be written as
%\begin{equation}
%  \eta= 
%  \qty(
%  \begin{array}{c}
%    0\\
%    0\\
%    0\\
%    \eta_0
%  \end{array}
%  )
%\end{equation}
%In other words, the existence of a covariantly constant spinor implies that the holonomy group 
%of the manifold is $SU(3)$.
\todoin{Integrate better}
The existence of a single covariantly constant spinor on the compact manifold can be reformulated
as a topological condition, namely that the holonomy group of the compact manifold is $SU(3)$.
A compact manifold of $SU(3)$ (local) holonomy is the definition of a Calabi-Yau manifold.
The holonomy group being a proper subgroup of $SU(3)$ is equivalent to having more than one covariantly 
constant spinor, which would lead to a larger degree of supersymmetry preserved.


We can also check that the 2-form $\mathbf {15}$ and the 3-form $\mathbf{20}$ decompositions contain a singlet, 
$\mathbf {15}\to \mathbf 8\oplus \mathbf 3\oplus \bar {\mathbf 3}\oplus \mathbf 1$ and 
$\mathbf {20}\to \mathbf 6\oplus \bar{\mathbf 6}\oplus\mathbf 3\oplus \bar {\mathbf 3}\oplus \mathbf 1\oplus \mathbf 1$,
so they are globally well defined.
We refer to the 2-form as $J$ and the 3-form as the holomorphic three-form $\Omega$.
Raising an index of $J$ we obtain an almost-complex structure, which satisfies $(J^2)^i_j=-\delta^i_j$.
For a particular point of the manifold, we can form a basis of complex coordinates $z^i$ from the real coordinates $x^i$,
as $z^1=x^1+ix^2$, $z^2=x^3+ix^4$ and $z^3=x^5+ix^6$,
in which $J=idz^i\otimes dz^i - i d\bar z^{\bar i}\otimes d\bar z^{\bar i}$.
If we can extend this particular form of $J$ to the neighborhood of any point, $J$ is said to be integrable
and the manifold is complex.
An integrable almost-complex structure is referred to as a complex structure. 
The integrability condition is equivalent to the Nijenhuis tensor 
\begin{equation}
  N^k_{ij}= J^l_i(\partial_l J^k_j - \partial_j J^k_l) - J_j^l (\partial_l J^k_i - \partial_i J^k_l)
\end{equation}
vanishing everywhere.
%It is possible to formulate an alternative definition of a complex manifold, as 
%\todoin{Complete following Ibanez}

It is useful to define with the aid of the metric the form $k=g_{i\bar j} dz^i \wedge d\bar z^{\bar j}$.
A manifold is Kähler if $dk=0$ and $k$ is then called the Kähler form.
It can be shown that the holonomy group being contained in $U(N)$ implies that the manifold is Kähler.

%The only  $U(3)$ invariants in the $\mathbf{6}$ representation of $SO(6)$ are the identity and
%$\bar I$.
%
%We can also form a tensor field on $K$ of the type $J^i_j(y)=g^{ik}(y) \bar\eta \Lambda_{kj} \eta(y)$.
%For each point $y$, we can consider $J^i_j$ as a matrix that acts on the tangent space, so $v^i \to J^i_j v^j$.
%In this sense, $J^i_j$ is a real, traceless and $SU(3)$ invariant matrix, which means that it must be proportional
%to $\bar I$.
%We had already seen that $\bar I = -I$, this an example of an almost-complex structure, which is 
%a tensor field $J$ that satisfies $J^2=-I$.
%
%If we employ complex coordinates, we can diagonalize $J$ so that the non-zero components are
%$J^a_b=i\delta^a_b$ and $J^{\bar a}_{\bar b}=-i\delta^{\bar a}_{\bar b}$. This particular choice
%is know as the canonical form.
%
%It is always possible to choose particular coordinates to bring $J$ to the canonical form at a particular 
%point.
%But in general, the canonical form will not hold at an open neighborhood of a point.
%If a manifold admits a set of coordinates (called local holomorphic coordinates) such that at every
%point, the canonical form holds for an open neighborhood, then the almost complex structure is integrable.
%
%The necessary and sufficient condition for integrability is that the Nijenhuis tensor
%
%\begin{equation}
%  N^k_{ij}= J^l_i(\partial_l J^k_j - \partial_j J^k_l) - J_j^l (\partial_l J^k_i - \partial_i J^k_l)
%\end{equation}
%
%vanishes.
%An integrable almost-complex structure is a complex structure and a manifold with a complex structure
%is a complex manifold.

\subsection{Cohomology}

It is useful to introduce some algebraic topology tools which we will use later on.

Let us consider a smooth manifold of dimension $d$. A differential $p$-form $\omega_p$ is $(0,p)$-rank tensor which has completely anti-symmetric components.
A $p$-form is expanded as a linear combination of the basis cotangent vectors $\qty{dx^\nu}_{\nu=1\ldots d}$  as
\begin{equation}
  \omega_p =\frac{1}{p!} \omega_{\nu_1\ldots\nu_p}dx^{\nu_1}\wedge \cdots  \wedge dx^{\nu_p} =
\frac{1}{p!} \omega_{\nu_1\ldots\nu_p}dx^{[ \nu_1}\otimes \cdots  \otimes dx^{\nu_p]},
\end{equation}
where the square brackets denote antisymmetrization.

The wedge product of a $p$-form $\omega_p$ and a $q$-form $\alpha_q$ is a $(p+q)$-form
\begin{equation}
  \omega_p \wedge \alpha_q  = \frac{1}{p!q!}\omega_{\nu_1\ldots\nu_p}
\alpha_{\mu_1\ldots\mu_q}dx^{\nu_1}\wedge \cdots  \wedge dx^{\nu_p}\wedge dx^{\mu_1}\wedge \cdots  \wedge dx^{\mu_q}.
\end{equation}

The exterior derivative of a $p$-form yields a $(p+1)$-form
\begin{equation}
  d\omega_p = \frac{1}{p!}\partial_\mu \omega_{\nu_1\ldots\nu_p}dx^\mu\wedge dx^{\nu_1}\wedge\cdots\wedge dx^{\nu_p}.
\end{equation}
A $p$-form whose exterior derivative vanishes is called closed and a $p$-form that is the exterior derivative
of a $(p-1)$-form is exact.

A fundamental property of the exterior derivative is Poincare's lemma, which states that for any differential form $\alpha$, $d(d\alpha)=0$ holds.
This can be rewritten as $d^2=0$. In other words, every exact form is closed.
We could ask ourselves if the inverse statement is true:
is every closed form exact?
The answer for an arbitrary manifold is no.%, there are closed forms that are not exact. 
This information is encoded in the $q$-th deRham cohomology group, which is formed by considering
the set of all closed $q$-forms defined on a manifold.
Since given a closed form $\omega$, we can always find another closed form by adding an exact form
$\omega' = \omega+d\alpha$, we take the equivalence relation that two forms are equivalent if
they differ by a closed form.
The $q$-th deRham cohomology group of a manifold $X$ is defined as the quotient
\begin{equation}
  H^q_d(X,\mathbb R)=\qty{\omega| d\omega=0}/\qty{\alpha|\alpha=d\beta}.
\end{equation}
The dimension of $H^q_d(X,\mathbb R)$ is the Betti number $b^q(X)$.
Only when $b^q(X)=1$, all closed $q$-forms on $X$ are exact.


We can easily make a generalization of the previous concepts to complex manifolds of complex dimension $n=d/2$.
Complexifying the basis $\qty{dx_{\mu}}_{\mu=1,\ldots,d} \to  \qty{dz_i,d\bar z_{\bar j}}_{i,j=1,\ldots,n}$,
we can consider tensors $\omega_{r,s}$ with $r$ holomorphic and $s$ anti-holomorphic indices so they can be written as 
\begin{equation}
  \omega_{r,s} = \omega_{\mu_1,\ldots\mu_r,\bar \nu_1,\ldots,\bar \nu_s} dz^{\mu_1}\wedge\cdots\wedge dz^{\mu_r}\wedge d\bar z^{\bar \nu_1}\wedge
  \cdots\wedge d\bar z^{\bar \nu_s}
\end{equation}
This allows to split the exterior derivative into holomorphic and anti-holomorphic derivatives $d=\partial+\bar\partial$. 
The complex equivalent of the deRham cohomology group is the Dolbeault cohomology group associated to $\bar \partial$ (it can be analogously be defined for $\partial$)
\begin{equation}
  H^{r,s}_{\bar \partial }(X,\mathbb C)=\qty {\omega | \bar \partial \omega= 0}/\qty{\alpha|\alpha=\bar\partial \beta}.
\end{equation}
We define the Hodge dual $\star$ of a $p$-form as the $(d-p)$-form
\begin{equation}
  \star \omega = \frac{1}{(n-p)!p!}\epsilon_{\mu_1\ldots\mu_n}\sqrt{|\det g|} g^{\mu_1 \nu_1}\ldots g^{\mu_p\ldots\nu_p}\omega_{\nu_1\ldots\nu_p}dx^{\mu_{p+1}}\wedge \ldots\wedge dx^{\mu_d}.
\end{equation}
This operation allows us to form the adjoint exterior derivative or codifferential $d^\dagger$, 
that maps $p$-forms into $(p-1)$-forms
\begin{equation}
  d^\dagger=(-1)^{np+n+1}\star d \star. 
\end{equation}
The codifferential is the adjoint of the exterior derivative with respect to the inner product
\begin{equation}
  \langle \omega,\omega'\rangle = \int_X \omega\wedge\omega',
\end{equation}
so that given a $p$-form $\omega$ and a $(p-1)$-form $\sigma$
\begin{equation}
  \langle \omega, d\sigma \rangle = \langle d^\dagger \omega, \sigma\rangle.
\end{equation}

The Laplacian can be generalized as $\Delta = dd^\dagger+d^\dagger d$ and
a harmonic form $\omega$ satisfies the Laplace equation $\Delta \omega=0$.

An important theorem is Hodge's decomposition, which states 
that a  $p$-form $\omega$ can be uniquely written in terms of a $(p-1)$-form $\beta$, 
a $(p+1)$-form $\gamma$ and a harmonic $p$-form $\omega'$
\begin{equation}
  \omega = d\beta + d^\dagger \gamma +\omega'
\end{equation}

If $\omega$ is a closed form, $\gamma$ vanishes so
\begin{equation}
  \omega = d\beta + \omega'.
\end{equation}
Identifying $\omega - d\beta= \omega'$ as an element of a cohomology class in $H_d^p(X,\mathbb R)$,
we can conclude that for every class belonging to $H_d^p(X,\mathbb R)$, there is an unique
harmonic $p$-form.

The dimension of $H^{r,s}_{\bar \partial }(X,\mathbb C)$ is known as the Hodge number, $h^{p,q}(X)$ and it is a topological invariant.
Thanks to the Hodge star, there is a relation between Hodge numbers $h^{p,q}=h^{n-p,n-q}$.
The fact that the manifold is Kähler also guarantees the symmetry $h^{p,q}=h^{q,p}$.
The decomposition of the deRham cohomology into Dolbeault cohomologies is given by
\begin{equation}
  H_d^p(X,\mathbb R)=\bigoplus_{r+s=p} H_{\bar \partial}^{r,s}(X,\mathbb C).
\end{equation}
In the case of Calabi-Yau manifolds, it also holds that $h^{s,0}=0$ if $1<s<n$, $h^{n,0}=h^{0,n}=1$.
If the manifold is connected, then $h^{0,0}=1$.

We can arrange the Hodge numbers into a Hodge diamond, which for a manifold of complex dimension
three would be
{\small \[
\arraycolsep=1.5pt
\begin{array}{ccccccc}
  &&&h^{00}&&&\\
 &&h^{10}&&h^{01}&&\\
 &h^{20}&&h^{11}&&h^{02}&\\
 h^{30}&&h^{21}&&h^{12}&&h^{03}\\
 &h^{20}&&h^{11}&&h^{02}&\\
 &&h^{10}&&h^{01}&&\\
  &&&h^{00}&&&
\end{array}
\]}

In the case of a Calabi-Yau three-fold
{\small \[
\arraycolsep=1.5pt
\begin{array}{ccccccc}
  &&&1&&&\\
 &&0&&0&&\\
 &0&&h^{11}&&0&\\
 1&&h^{21}&&h^{21}&&1\\
 &0&&h^{11}&&0&\\
 &&0&&0&&\\
  &&&1&&&
\end{array}
\]}

A very fruitful relation between manifolds is mirror symmetry, which associates topological
properties between different Calabi-Yau manifold.
A realization of mirror symmetry for a Calabi-Yau three-fold $X$ is that there exists a mirror manifold $Y$
such that
\begin{equation}
  h_{1,1}(X)=h_{2,1}(X) \qquad and \qquad h_{2,1}(X)=h_{1,1}(Y)
\end{equation}
Thus, we can calculate some properties on $X$ and immediately obtain information on its mirror manifold.

\subsection{Homology}

A very related construction to cohomology is homology.
The basic element of homology is the $p$-chain $a_p$, which in the simplest formulation is the
formal sum of $p$-dimensional submanifolds $N_p^k$ (possibly with boundary)
\begin{equation}
  a_p = \sum_k c_k N_p^k
  \label{eq:exp}
\end{equation}
where $c_k$ are real numbers.
$p$-forms that have no boundary are called $p$-cycles. The boundary operator $\partial$ satisfies $\partial^2$.
The homology group $H_q(X,\mathbb R)$ is defined as the quotient space of $q$-cycles modulo $q$-dimensional
boundaries
\begin{equation}
  H_q(X,\mathbb R) =\qty{a| \partial a=0}/\qty{b|b=\partial a}
\end{equation}
The dimension of $H_q(X,\mathbb R)$ is $h_q(X)$.
It is sometimes convenient to consider the coefficients $c_k$ of the expansion \eqref{eq:exp} to
be integers, in that case the associated homology group is $H_p(X,\mathbb Z)$.

It can be seen that homology resembles cohomology replacing chains by forms, the boundary operator
by the exterior derivative and cycles by closed forms.
In fact, they are algebraic duals, in the sense that integration of a $p$-form over a $p$-chain
defines and isomorphism between $H_q$ and $H^q_d$ in the case of compact manifolds. 
This implies that the dimensions of both group coincide $h_q=h^q$.

It is useful to generalize the concept of how many times to lines intersect to the case of
$p$-cycles.
The intersection number is a topological invariant, so it depends on the homology
classes of $H(X,\mathbb Z)$ only.
Given a $p$-cycle $a_p$ and a $d-p$-cycle $b_{d-p}$ the intersection number is defined as
\begin{equation}
  [a_p] [b_{d-p}] = \int_X \delta(a_p)\wedge \delta(b_{d-p})
\end{equation}
where Dirac delta function satisfies
\begin{equation}
  \int_{a_p} B_p = \int_X B_p \wedge \delta(a_p).
\end{equation}

\subsection{Moduli space}

Starting from a particular choice of metric $g$ on a Calabi-Yau manifold $X$, we could try to determine
which deformations of the metric still preserve the Calabi-Yau condition.
These deformations of the metric are known as moduli and play an important role in the physics of compactifications.
We will restrict our discussion to Calabi-Yau manifolds of complex dimension three.
An arbitrary deformation of the metric will consist of those with pure 
indices $g_{ij}dz^i dz^j$ and those with mixed indices $g_{i\bar j}dz^i dz^{\bar j}$.
In order to preserve the Calabi-Yau condition they must lead to a vanishing Ricci tensor, $R_{i\bar j}=0$.
This constraint implies that:

 A deformation of the type $g_{ij}dz^i \wedge dz^j$ must be harmonic, so it can be identified with an unique element of a cohomology class in $H^{1,1}$, the Kähler form.
 If we write the Kähler form in terms of the basis elements $\qty{t_a}_{a=1,\ldots,h_{1,1}}$
\begin{equation}
k=\sum_{a=1}^{h_{1,1}}t_a \omega_a
\end{equation}
the $h_{1,1}$ real parameters $t_a$ are the Kähler moduli of the manifold.
The Kähler form is employed to calculate the volume of a Calabi-Yau manifold of complex dimension three as
$\int k\wedge k\wedge k$, since $k\wedge k\wedge \wedge k$ has the same rank as the
volume form, which is unique up to a proportionality constant.

Deformations of the type $\Omega_{ijk}g^{k\bar k}\delta_{\bar k\bar l}dz^i\wedge dz^j \wedge d\bar z^{\bar l}$ must be a harmonic form belonging to a cohomology class in $H^{2,1}$.
These deformations correspond to deformations of the complex structure, 
since the choice of a complex structure is related to a $(2,1)$-form  $J_{ij\bar k}= \Omega_{ijl}J^l_{\bar k}$ 
obtained from the holomorphic three-form.
%We can expand the $(2,1)$-form in a basis $\qty{s_a}_{a=1,\ldots,h_{2,1}}$
%\begin{equation}
%J=\sum_{a=1}^{h_{2,1}}s_a \sigma_a
%\end{equation}
There are $h_{2,1}$ complex parameters associated to the choice of the complex structure, which are called the complex structure moduli of the manifold.
They determine the volume of 3-cycles $\Pi$ in the compact space through $\Omega_3$ 
\begin{equation}
  \mathrm{Vol}(\Pi)=\int_\Pi \Omega_3.
\end{equation}

In conclusion, a Calabi-Yau metric is determined uniquely by the Kähler form and the holomorphic three-form.
The former leads to $h_{1,1}$ real parameters while the latter requires $h_{2,1}$ complex parameters.


\subsection{Type IIA spectrum on Calabi-Yau manifolds}
In order to compute the 4-dimensional massless spectrum of type IIA theory on a Calabi-Yau, we consider the 
Kaluza-Klein dimensional reduction.
This consists in choosing an energy scale at which the compactification resides (the KK-scale)
and then studying the effective four-dimensional theory at energies below the KK-scale.
In practice, this corresponds to taking the KK-scale relatively large (or equivalently taking the associated radius of the compact space very small). 

The simplest example of KK reduction is based on a free scalar field $\phi(x^M)$ in ten dimensions.
We first apply its Fourier expansion in terms of the eigenvectors $\phi_k(x^m)$ of the Laplace operator in the internal space  with eigenvalues $\lambda_k$
\begin{equation}
  \phi(x^M)= \sum_k \phi_k(x^\mu)\phi_k(x^m)
\end{equation}
where the dimension of the mode is determined by the argument, $x^\mu$ for the 4-dimensional Minkowski space and $x^m$ for the compact space.
The masslessness condition of $\phi(x^M)$ implies that
\begin{equation}
 \Box \phi(x^\mu) - \lambda_k \phi(x^\mu)=0 
\end{equation}
This equation permits us to identify $\lambda_k$  as the squared mass of the 4-dimensional $\phi(x^\mu)$ field.
Thus, the number of massless scalar fields is given by the number of solutions of $\Box \phi(x^\mu)=0$ which in the
case of compact manifolds is one.
We conclude that a 10-dimensional scalar field leads to a massless scalar field in 4-dimensions (in addition, the is a tower of KK modes).

Our next example is the KK reduction of a $p$-form $C_p$ with the expansion 
\begin{equation}
  C_p=\sum_{k,q} c_q^k(x^m)\wedge C^k_{p-q}(x^\mu)
\end{equation}
Massless 4-dimensional $(p-q)$-form fields correspond to internal modes that satisfy $dc_q=d^\dagger c_q=0$, so 
$c_q$ is a harmonic form.
Since there is a single harmonic $q$-form in each $q$-cohomology class, 
the number of 4-dimensional massless $(p-q)$-forms arising from a $p$-form is the dimension of the $H_q$ cohomology group, the 
Betti number $b_q$.
\todo{Check if correction is needed.}

In the case of a Calabi-Yau manifold, from the relation of the Betti numbers with the Hodge numbers,
we determine $b_0=h_{0,0}=1$, $b_1=h_{1,0}+h_{0,1}=0$, $b_2=h_{1,1}+h_{2,0}+h_{0,2}=h_{1,1}$ and
$b_3=h_{3,0}+h_{0,3}+h_{2,1}+h_{1,2}=2h_{2,1}+1$.
Thus, $c_1$ leads to a 4-dimensional 1-form, 
$B_2$ leads to a 2-form and  $h_{1,1}$ scalar fields 
and $c_3$ leads to a 3-form (although it is not dynamical), $h_{1,1}$ 1-forms and $2h_{2,1}+2$ scalar fields.

The KK reduction of the 10-dimensional metric is applied considering its components separately:
\begin{itemize}
  \item The $G_{\mu\nu}$ components correspond to scalar fields in the internal space satisfying 
    the Laplace equation and whose solution is unique for compact spaces.
    Thus, a 10-dimensional graviton reduces to a 4-dimensional graviton.
  \item The $G_{\mu m}$ components would correspond to 4-dimensional vector bosons, associated
    to 6-dimensional vector fields in the compact space. 
    The masslessness condition of the 4-dimensional field would imply that the 6-dimensional 
    vectors are Killing vectors associated to continuous isometries of the compact space, which
    in the case of Calabi-Yau manifolds are non-existent.
    As a consequence, the $G_{\mu m}$ components do not lead to any massless fields in 4 dimensions.
  \item The $G_{m n}$ components reduce to 4-dimensional scalar fields associated to the moduli of the internal space,
    whose vev determine the geometry of the internal space.
    In the case of Calabi-Yau manifolds, we have seen that there are $h_{21}$ real scalar fields and
    $h_{11}$ complex scalar fields.

%$\qty{\omega_P}_{P=1,\ldots,h_{1,1}}}$ 
%$\qty{\alpha_K, \beta^K}_{K=1,\ldots,h_{2,1}}$
%
%
%$\int \alpha_K \wedge \beta^L = \delta_K^L$
\end{itemize}

Having seen how the bosonic fields of type IIA behave under KK reduction, we proceed to describe the
massless spectrum of type IIA theory compactified on a Calabi-Yau manifold.

In order to fill in the supermultiplets of 4-dimensional $\mathcal N=2$ supersymmetry,
we must combine scalar fields arising from the dilaton $\phi$, $p$-forms and the geometric
moduli into complex scalar fields.
The spectrum is arranged as follows:

A single supergravity multiplet, composed of a graviton $G_{\mu\nu}$, a gauge boson arising from the KK reduction of $B_2$ and two gravitinos $\psi$ with opposite chiralities. 

$h_{1,1,}$ vector multiplets, composed of a gauge boson that arises from $c_3$, a complex scalar (obtained by combining the Kähler moduli $t_a$ and the scalar field $B_0$ associated to $B_2$ into $B_2+it_a$) and two Majorana fermions.

$h_{2,1}$ vector hypermultiplets composed of two complex scalars (obtained by combining the complex structure moduli with the scalar fields associated to mixed index components of $c_3$) and two left-handed fermions.

A single vector hypermultiplet composed of two complex scalars (obtained by combining the dilaton, a scalar field associated to $B_2$
\footnote{The KK reduction of the 10-dimensional 2-form $B_2$ leads to a 4-dimensional 2-form $b_2$.
We can then define a scalar field $\tilde b$ as the dual $d \tilde b = \star d b_2$.} and the scalars that arise from pure index components of $c_3$) and two left-handed fermions.

\section{Type IIA on orientifold projections}

\subsection{Generalities of orientifolds}
If we compactify a type II string theory on a Calabi-Yau manifold, we  obtain a four-dimensional
$\mathcal N=2$ supersymmetric theory.
This degree of supersymmetry does not allow for chiral fermions, so Calabi-Yau compactifications
of type II theories have no straightforward application in the context of model building.
An option to reduce the supersymmetry to $\mathcal N=1$  is to apply the orientifold
projection, which consists in modding out the action of $\Omega R$,
where $\Omega$ is the worldsheet parity, so strings become unoriented, and
$R$ is a particular $\mathbb Z_2$ symmetry of the compact six-dimensional space
In type IIA string theory we define $R=\mathcal R (-1)^{F_L}$.
$\mathcal R$ satisfies the condition that it is an involution (squares to the identity) and 
acts anti-holomorphically on the complex coordinates of the internal space ($\mathcal R: z_i \to \bar z_i$).
This implies that the Kähler and the holomorphic three-form transform as $J\to -J$ and $\Omega_3 \to \bar \Omega_3$.
$F_L$ is an operator that counts the number of left-moving fermions.

The fixed points under $\mathcal R$ define the orientifold planes in the model and are denoted
as O$p$-planes, where $p$ is the spatial dimension.
In type IIA theory, the only consistent choice are O$6$-planes,
which span the entire four-dimensional Minkowski space and wrap a compact $3$-cycle on the internal space.

In order to have a stable compactification, we expect all RR and NSNS charges to vanish.
Furthermore, RR tadpole cancellation implies that the 4-dimensional theory is free of non-abelian gauge anomalies.
O$6$-planes carry RR charge, so in order to eliminate RR tadpoles we must also introduce D$6$-branes,
which carry opposite charge.
It is important to note that D$6$-branes do not need to wrap the same 3-cycles as the O$6$-planes
to remove RR tadpoles.

\subsection{D-branes on Calabi-Yau manifolds}

In order to obtain stable D6-brane configurations on a type IIA theory compactified on a Calabi-Yau manifold, 
we impose that they wrap around volume minimizing 3-cycles on the compact space, so that their tension is minimized as well.
The volume minimizing condition means that the branes must wrap special Lagrangian 3-cycles in the internal space.
Special Lagrangian 3-cycles $\Pi$ are defined by
\begin{equation}
  k|_\Pi = 0 , \qquad \Im (e^{-i\phi}\Omega_3)|_\Pi=0
\end{equation}
for some real $\phi$, where $k$ is the Kähler two-form and $\Omega_3$ the holomorphic three-form.
The $e^{-i\phi}\Omega_3$ is referred to as a calibration and the special Lagrangian is calibrated with respect to it.%, for every choice of $\phi$.
The volume of the special Lagrangian 3-cycle is
\begin{equation}
  \mathrm{Vol}(\Pi)=\int_\Pi \Re(e^{-i\phi}\Omega_3)
\end{equation}

D6-branes wrapped around a special Lagrangian cycle are guaranteed to preserve 4-dimensional $\mathcal N=1$ supersymmetry. 
This preserved supersymmetry coincides with the same supersymmetry preserved by the O$p$-planes only if $\phi=0$.

The open string spectrum of stacks of $N_a$ D$6_a$-branes wrapping special Lagrangian 3-cycles $\Pi_a$ 
can be classified into two sectors: strings that stretch from one stack to itself and those that stretch between to different stacks, $6_a$ and $6_b$.

Strings that stretch over $6_a$ lead to $U(N_a)$ vector multiplets of 4-dimensional $\mathcal N=1$ supersymmetry.
There are also $b_1(\Pi_a)$ chiral multiplets in the adjoint representation,
which are composed of the internal components of the gauge fields along $\Pi_a$ combined with 
the geometric moduli of the $3$-cycle, and their fermion superpartners.

Strings that stretch between $6_a$ and $6_b$ lead to $I_{ab}=[\Pi_a][\Pi_b]$ chiral fermions, where $I_{ab}$ 
is the intersection number between $3$-cycles. These fermions transform in the $(\mathbf{N_a},\mathbf{\bar N_b})$ representation.
There are also massless scalar fields if the intersection preserves supersymmetry.

\subsection{Orientifold compactifications with intersecting D-branes}

We consider $N_a$ D$6$-branes that wrap 3-cycles $\Pi_a$ and whose image under the orientifold
projection wrap the 3-cycles $\Pi_{a'}$.
The condition that D$6$-branes preserve the same $\mathcal N=1$ supersymmetry as the O$6$-planes $\Pi_{O6}$
is that the local relative angles between them obey
\begin{equation}
  \theta_1 + \theta_2 + \theta_3 = 0
\end{equation}

If D$6$-branes do not coincide with their mirror images, the light spectrum of the model consists of:
\begin{itemize}
  \item $U(N_a)$ gauge bosons arising from non-intersecting D$6$-branes.
  \item $I_{ab}$ fermions in the representation $(\mathbf{N_a},\mathbf{\bar N_b})$ arising from the
    intersection of two different D$6$-branes.
  \item $I_{ab'}$ fermions in the representation $(\mathbf{N_a},\mathbf{N_b})$ arising from the intersection of 
    a D$6$-brane with the mirror of a different D$6$-brane.
  \item $1/2([\Pi_a][\Pi_{a'}]+[\Pi_a][\Pi_{O6}])$ fermions in the anti-symmetric representation $({\tiny\yng(1,1)},\mathbf 1)$ and
    $1/2([\Pi_a][\Pi_{a'}]-[\Pi_a][\Pi_{O6}])$ fermions in the symmetric representation $({\tiny\yng(2)},\mathbf 1)$ which
    arise from the intersection of a D$6$-brane with its own mirror.
\end{itemize}

The condition for RR tadpole cancellation imposes a topological restriction, namely, 
that the sum of the three-cycles wrapped by the D-branes and their orientifold images
has to combine with the O$6$-plane three-cycle into the trivial cycle in homology
\begin{equation}
  \sum_a N_a \qty([\Pi_a] + [\Pi_{a'}]) - 4 [\Pi_{O6}]=0.
\end{equation}

\subsection{Effective action of D-branes on Calabi-Yau orientifolds}

%\begin{equation}
%\begin{aligned}
%C_1 = C_1(x)\\
%B_2 = b_2(x)+b^p \omega_p \\
%C_3 = A_1^P(x)\wedge \omega_P + C^K(x)\alpha_K -\tilde C_K(x)\beta^K
%\end{aligned}
%\end{equation}

We recall that the action of a D$p$-brane contains the DBI term \eqref{eq:dirac}
\begin{equation}
  S_{DBI} = -\mu_p \int_{D_p} e^{-\phi}\sqrt{\det (G+B-2\pi \alpha' F)}
\end{equation}
which reduces to the Yang-Mills action for small values of $\alpha'$.
In the case of compactification on a Calabi-Yau orientifold, the gauge coupling  constant is given
in terms of the volume of the special-Lagrangian three-cycles along the internal space
\begin{equation}
  \frac{1}{g^2}=e^{-\phi}\frac{(\alpha')^{-3/2}}{(2\pi)^{4}}\mathrm{Vol}(\Pi_{3}).
\end{equation}

%\begin{equation}
%  S = \mu_p \int_{D_p} \sum_q \Tr e^{2\pi \alpha' F-B}+\cdots
%\end{equation}
%

%\begin{equation}
%  f_a = \frac{(\alpha')^{3/2}}{(2\pi)^4}\qty[e^{-\phi}\int_{\Pi_a}\Re(e^{-i\phi_a}\Omega_3)+ i\int_{\Pi_a}c_3]
%\end{equation}
