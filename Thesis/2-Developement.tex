\chapter{Type IIA compactifications}

In the following section we motivate the requirement that additional dimensions are compactified
over a Calabi-Yau manifold.

Since Type IIA requires nine spatial dimensions but we only observe three, we need to compactify six of 
them over a small region.
We assume that the manifold $M$ is factorizable into a four-dimensional maximally symmetric space-time $T$ and a six-dimensional compact space $K$,
$M =  T\times K$.

Type IIA string theory on 10 dimensional flat space has a large degree of supersymmetry,
but the compactification choice can either preserve some degree of supersymmetry in four dimensions or remove it completely.
We will consider compactifications over an internal manifold that leave some supersymmetry in four dimensions intact.
A historical motivation for this choice is that they provide a nice way to obtain realistic particle 
physics models. 
In particular, we will see that a $\mathcal N=1$ supersymmetric theory allows for chiral fermions in four dimensions, while field theories with a higher mumber of supersymmetry in four dimensions do not.
In addition, supersymmetric configurations are easier to study before tackling more general compactifications.
Indeed, supersymmetric compactifications of string theory allow  for stable dimensional reductions, whose higher-dimensional corrections can be systematically studied.

The algebra of a N=1 supersymmetric theory in four-dimensional Minkoswki spacetime is an extension of the Poincare-algebra by adding
 supersymmetry generators which satisfy specific anti-commutation relations, instead of commutation relations. 

\todoin{Reformulate. We want to obtain a condition for a susy vacuum in 4d. We can obtain a classical field theory constraint.}
A conserved charge $Q$ associated to an unbroken supersymmetry annihilates the vacuum $\ket{\Omega}$,
so $Q\ket{\Omega}=0$.
This in turn means that for any operator $U$, $\ev{ \{Q,U\} }{\Omega}=0$.
If $U$ is a fermionic operator, we derive that the variation of the operator under the supersymmetry
transformation is $\delta U = \{ Q, U\}$.
Taking this as the classical limit, $\delta U = \ev{\delta U}{\Omega}$.
Thus, we conclude that at the classical level $\delta U =\ev{ \{Q,U\} }{\Omega}= 0$ for any fermionic field $U$.

We now consider the SUGRA theory of type IIA string theory and
the condition that some fourd-supersymmetry remains.
In the same way that a translation generated by the momentum operator is parametrized by a vector and
a rotation is a parametrized by an antisymmetric tensor, a supersymmetry transformation generated
by $Q_\alpha$ is parametrized 
by a spinor $\eta_\alpha$.
The variation of the gravitino field under a supersymmetry transformation is
\begin{equation}
  \delta \psi_M = D_M \eta + \mathrm{(fluxes)}
\end{equation}

Where $D_M$ is the covariant derivative on $M$.
Supersymmetry preservation means that all variations must be zero. 
We assume that all fluxes vanish.
This leads to the constraint that $\eta$ is a covariantly constant spinor
\begin{equation}
  \delta \psi_M = D_M \eta = 0
%  \delta \xi^a &= -\frac{1}{4g\sqrt \phi} \Gamma^{MN} F^a_{MN} \eta 
  \label{eq:cov}
\end{equation}

%The equation implies that there exists a spinor $\eta$ such that  $[D_M,D_N]\eta=R_{MNPQ} \Gamma^{PQ} \eta=0$.
If we particularize this equation to the four-dimensional space-time  $T$, which is a maximally symmetric space, 
it imposes that $T$ is Minkowski space and thus, $\eta$ only depends on the compact coordinates.

To study the implication of this equation to the four-dimensional space-time $T$, we employ the 
fact that $T$ is maximally symmetric, so we can decompose  the metric as
\begin{equation}
  ds^2=e^{2A(y)}\tilde {g}_{\mu\nu} dx^\mu dx^\nu + g_{mn} dy^m dy^n, \qquad  \mu=0,1,2,3 \quad m=1,\dots,6
\end{equation}

where $x^\mu$ are the compact coordinates, $y^m$ the internal coordinates and $\tilde{g}_{\mu\nu}$ can 
be either the de Sitter, anti-de Sitter or the Minkowski metric in four dimensions.
%\todoin{\url{https://groups.google.com/forum/#!topic/sci.physics.research/rrBoIXk9Rw0}}

Particularizing to the space-time components, equation \eqref{eq:cov} can be written as
\begin{equation}
  \widetilde{\nabla}_\mu \eta + \frac{1}{2}\qty(\tilde{\gamma}_\mu \gamma_5 \otimes \slashed \nabla A) \eta=0
\end{equation}
where $\widetilde{\nabla}$ is the derivative with respect $\tilde g_{\mu\nu}$.
\todoin{What is the gamma}

\todoin{Should I explain where this comes from? integrability}
We then obtain
\begin{equation}
  \qty[\widetilde \nabla_\mu,\widetilde \nabla_\nu]\eta=\frac{1}{2}\qty(\nabla_m A)\qty(\nabla^m A)\gamma_{\mu\nu}\eta
\end{equation}

\begin{equation}
  \qty[\widetilde \nabla_\mu,\widetilde \nabla_\nu]\eta=\frac{1}{4}\tilde R_{\mu\nu\lambda\rho} \gamma^{\lambda\lambda}\eta
\end{equation}

Since for a maximally symmetric space, the Riemman tensor is $R_{\mu\nu\lambda\rho}=k(g_{\mu\lambda}g_{\nu\rho}-g_{\mu\rho}g_{\nu\lambda})$.
Combining the previous two equations and inverting $\gamma^{\mu\nu}$, we obtain the condition
\begin{equation}
  k + \nabla_m A \nabla^m A =0
\end{equation}

Constant value etc

The existence of a single covariantly constant spinor on the compact manifold can be reformulated
as a topological condition, namely that the holonomy group (whose precise definition is given in the next chapter) of the compact manifold is $SU(3)$.
A compact manifold of $SU(3)$ (local) holonomy is the definition of a Calabi-Yau manifold.
The holonomy group being a proper subgroup of $SU(3)$ is equivalent to having more than one covariantly 
constant spinor, which would lead to a larger degree of supersymmetry preserved.
%%TODO: topological vs differential. group structure vs holonomy. existence vs cov. const.

The existence of a covariantly constant spinor implies  for a TIIA theory that there are two four-dimensional supersymmetry parameters and therefore, $\mathcal N =2$.

\section{Type IIA on Calabi-Yau manifolds}
We examine more closely what the existence of a covariantly constant spinor field implies on the compact space. 

Let us consider a Riemannian manifold $K$ of dimension six with a spin connection $\omega$, which 
is in general a $SO(6)$ gauge field.
If we parallel transport a field $\psi$ around a contractible closed curve $\gamma$, the field becomes
$\psi'=U\psi$ where $U=\mathcal P e^{\int_\gamma dx \omega}$ and $\mathcal P$ denotes the path ordering of 
the exponential.
The set of transformation matrices associated to all possible loops form the holonomy group of the manifold, 
which must be a subgroup of $SO(6)$.

%\todo{What does the existence of a non-trivial spinor tell us}
A covariantly constant spinor is left unchanged when parallel transported along a contractible
closed curve, so the holonomy matrices of a manifold that admits a covariantly constant spinor 
must satisfy $U\eta = \eta$.
Taking into account the Lie algebra isomorphism $\mathfrak{so}(6)\simeq \mathfrak{su}(4)$ we identify the positive
(negative)-chirality spinors of $SO(6)$ with the fundamental $\mathbf 4$ ($\mathbf {\bar 4}$)
of $SU(4)$.
Let us consider that $\eta$ is a positive chirality spinor, so it transforms according with the 
$\mathbf 4$ of $SU(4)$.
In order to have a covariantly constant spinor, the holonomy group must be such that the $\mathbf 4$
representation decomposes into a singlet.
This decomposition is achieved if the holonomy group is $SU(3)$ so that
 \begin{align}
  SO(6)  &\to SU(3)\\
  \mathbf 4 &\to \mathbf 3 \oplus \mathbf 1
\end{align} 
%Then, $\eta$ is left invariant under $SU(3)$ transformations and can be written as
%\begin{equation}
%  \eta= 
%  \qty(
%  \begin{array}{c}
%    0\\
%    0\\
%    0\\
%    \eta_0
%  \end{array}
%  )
%\end{equation}
%In other words, the existence of a covariantly constant spinor implies that the holonomy group 
%of the manifold is $SU(3)$.


We can also check that the 2-form $\mathbf {15}$ and the 3-form $\mathbf{20}$ decompositions contain a singlet, 
$\mathbf {15}\to \mathbf 8\oplus \mathbf 3\oplus \bar {\mathbf 3}\oplus \mathbf 1$ and 
$\mathbf {20}\to \mathbf 6\oplus \bar{\mathbf 6}\oplus\mathbf 3\oplus \bar {\mathbf 3}\oplus \mathbf 1\oplus \mathbf 1$,
so they are globally well defined.
We refer to the 2-form as $J$ and the 3-form as the holomorphic three-form $\Omega$.
Raising an index of $J$ we obtain an almost-complex structure, which satisfies $(J^2)^i_j=-\delta^i_j$.
For a particular point of the manifold, we can form a basis of complex coordinates $z^i$ from the real coordinates $x^i$,
as $z^1=x^1+ix^2$, $z^2=x^3+ix^4$ and $z^3=x^5+ix^6$,
in which $J=idz^i\otimes dz^i - i d\bar z^{\bar i}\otimes d\bar z^{\bar i}$.
If we can extend this particular form of $J$ to the neighborhood of any point, $J$ is said to be integrable
and the manifold is complex.
An integrable almost-complex structure is referred to as a complex structure. 
The integrability condition is equivalent to the Nijenhuis tensor 
\begin{equation}
  N^k_{ij}= J^l_i(\partial_l J^k_j - \partial_j J^k_l) - J_j^l (\partial_l J^k_i - \partial_i J^k_l)
\end{equation}
vanishing everywhere.
It is possible to formulate an alternative definition of a complex manifold, as 

It is useful to define with the aid of the metric the form $k=g_{i\bar j} dz^i \wedge d\bar z^{\bar j}$.
A manifold is Kähler if $dk=0$ and $k$ is then called the Kähler form.
It can be shown that the holonomy group being contained in $U(N)$ implies that the manifold is Kähler.

%The only  $U(3)$ invariants in the $\mathbf{6}$ representation of $SO(6)$ are the identity and
%$\bar I$.
%
%We can also form a tensor field on $K$ of the type $J^i_j(y)=g^{ik}(y) \bar\eta \Lambda_{kj} \eta(y)$.
%For each point $y$, we can consider $J^i_j$ as a matrix that acts on the tangent space, so $v^i \to J^i_j v^j$.
%In this sense, $J^i_j$ is a real, traceless and $SU(3)$ invariant matrix, which means that it must be proportional
%to $\bar I$.
%We had already seen that $\bar I = -I$, this an example of an almost-complex structure, which is 
%a tensor field $J$ that satisfies $J^2=-I$.
%
%If we employ complex coordinates, we can diagonalize $J$ so that the non-zero components are
%$J^a_b=i\delta^a_b$ and $J^{\bar a}_{\bar b}=-i\delta^{\bar a}_{\bar b}$. This particular choice
%is know as the canonical form.
%
%It is always possible to choose particular coordinates to bring $J$ to the canonical form at a particular 
%point.
%But in general, the canonical form will not hold at an open neighborhood of a point.
%If a manifold admits a set of coordinates (called local holomorphic coordinates) such that at every
%point, the canonical form holds for an open neighborhood, then the almost complex structure is integrable.
%
%The necessary and sufficient condition for integrability is that the Nijenhuis tensor
%
%\begin{equation}
%  N^k_{ij}= J^l_i(\partial_l J^k_j - \partial_j J^k_l) - J_j^l (\partial_l J^k_i - \partial_i J^k_l)
%\end{equation}
%
%vanishes.
%An integrable almost-complex structure is a complex structure and a manifold with a complex structure
%is a complex manifold.

%\todoin{Coordinate definition of complex manifold}

\subsection{Cohomology}

It is useful to introduce some algebraic topology tools.

A differential $p$-form $\omega_p$ is $(0,p)$-rank tensor which has completely anti-symmetric components.
A $p$-form is expanded as a linear combination of the basis $\qty{dx^\nu}_{\nu=1\ldots p}$ 
\begin{equation}
  \omega_p =\frac{1}{p!} \omega_{\nu_1\ldots\nu_p}dx^{\nu_1}\wedge \cdots  \wedge dx^{\nu_p} =
\frac{1}{p!} \omega_{\nu_1\ldots\nu_p}dx^{[ \nu_1}\otimes \cdots  \otimes dx^{\nu_p]}
\end{equation}
where the square brackets denote antisymmetrization.

The wedge product of a $p$-form $\omega_p$ and a $q$-form $\alpha_q$ is a $(p+q)$-form
\begin{equation}
  \omega_p \wedge \alpha_q  = \frac{1}{p!q!}\omega_{\nu_1\ldots\nu_p}
\alpha_{\mu_1\ldots\mu_p}dx^{\nu_1}\wedge \cdots  \wedge dx^{\nu_p}\wedge dx^{\mu_1}\wedge \cdots  \wedge dx^{\mu_p} 
\end{equation}

The exterior derivative of a $p$-form yields a $(p+1)$-form
\begin{equation}
  d\omega_p = \frac{1}{p!}\partial_\mu \omega_{\nu_1\ldots\nu_p}dx^\mu\wedge dx^{\nu_1}\wedge\cdots\wedge dx^{\nu_p}
\end{equation}

A $p$-form whose exterior derivative vanishes is called closed and a $p$-form that is the exterior derivative
of a $(p-1)$-form is exact.

A fundamental property of the exterior derivative is Poincare's lemma, which states that for any diffential form $\alpha$, $d(d\alpha)=0$ holds.
This can be rewritten as $d^2=0$. In other words, every exact form is closed.
We could ask ourselves if the inverse statement is true, 
%that is, if a form $\omega$ satisfies $d\omega$, can we find a form $\beta$ such that $\omega=d\beta$?
Is every closed form exact?
The answer for an arbitrary manifold is no.%, there are closed forms that are not exact.
This information is encoded in the $q$-th deRham cohomology group, which is formed by considering
the set of all closed $q$-forms defined on a manifold.
Since given a closed form $\omega$, we can always find another closed form by adding an exact form
$\omega' = \omega+d\alpha$, we take the equivalence relation that two forms are equivalent if
they differ by a closed form.
The $q$-th deRham cohomology group is defined as the quotient
\begin{equation}
  H^q_d(X,\mathbb R)=\qty{\omega| d\omega=0}/\qty{\alpha|\alpha=d\beta}
\end{equation}

Dimension Betti number

DOLBEAULT COHOMOLOGY

We can easily make a generalization of the previous concepts to complex manifolds.
$\bar\partial$
\begin{equation}
  H^{r,s}_{\bar \partial }(X,\mathbb C)=\qty {\omega | \bar \partial \omega= 0}/\qty{\alpha|\alpha=\bar\partial \beta}
\end{equation}

HODGE DECOMPOSITION / HODGE NUMBERS


\begin{equation}
  \omega = d\beta + d^\dagger \gamma +\omega'
\end{equation}

\begin{equation}
  \omega = d\beta + \omega'
\end{equation}

The dimension of $H^{p,q}$ is know as the Hodge number, $h_{p,q}$ and it is a topological invariant.
For Calabi-Yau manifolds, $h_{d,0}=1$
$h_{p,q}=h_{d-p,d-q}$ mirror symmetry

Hodge symmetry $h_{p,q}=h_{q,p}$

\subsection{Homology}

A very related construction to cohomology is homology.
The basic element of homology is the $p$-chain

algebraic dual, in the sense that


Poincaré duality 


\subsection{Moduli space}
%
%
%\todoin{Calabi-Yau definition: SU(n) global holonomy<->non-vanishing n-form -> vanishing first Chern class <->vanishing Ricci curvature<->local 
%SU(n) holonomy. If simply connected, both definitions are equivalent.}
%We start from a complex manifold with a metric st. it is a Calabi-Yau manifold.
%We then deform it's metric while it remains CY
%
%
%Given a Calabi-Yau manifold with a particular complex structure and metric, we could ask ourselves
%when the deformations

A Calabi-Yau manifold is determined uniquely by the Kähler form and the holomorphic three-form.

The volume form of a manifold is unique up to a proportionality constant, since $J\wedge J\wedge J$ has
the same rank as the compact space
\begin{equation}
  \mathrm{Vol}(X)=\int k \wedge k \wedge k
\end{equation}

\begin{equation}
k=\sum_{a=1}^{h_{1,1}}t_a \omega_a
\end{equation}

The $h_{1,1}$ real parameters $t_a$ are the Kähler moduli of the manifold.


\begin{equation}
I=\sum_{a=1}^{h_{2,1}}s_a \sigma_a
\end{equation}

$I_{ij\bar k}= \Omega_{ijl}I^l_{\bar k}$
The $h_{2,1}$ complex parameters $s_a$ are the complex structure moduli of the manifold
and determine the volume of 3-cycles $\Pi$ in the compact space

\begin{equation}
  \mathrm{Vol}(\Pi)=\int_\Pi \Omega_3
\end{equation}
\todoin{Complete}


\subsection{Type IIA spectrum on CY}
In order to compute the 4-dimensional massless spectrum of type IIA theory on a Calabi-Yau, we consider the 
limit where the size of compact space tends to zero (Kaluza-Klein dimensional reduction).

The simplest example of KK reduction is based on a free scalar field $\phi(x^M)$ in ten dimensions.
We first apply its Fourier expansion in terms of the eigenvectors $\phi_k(x^m)$ of the Laplace operator in the internal space  with eigenvalues $\lambda_k$
\begin{equation}
  \phi(x^M)= \sum_k \phi_k(x^\mu)\phi_k(x^m)
\end{equation}
where the dimension of the mode is determined by the argument, $x^\mu$ for the 4-dimensional Minkowski space and $x^m$ for the compact space.
The massless condition of $\phi(x^M)$ implies that
\begin{equation}
 \Box \phi(x^\mu) - \lambda_k \phi(x^\mu)=0 
\end{equation}
\todoin{Is phi massless in 10d?}
This equation permits us to identify $\lambda_k$  as the squared mass of the 4-dimensional $\phi(x^\mu)$ field.
Thus, the number of massless scalar fields is given by the number of solutions of $\Box \phi(x^\mu)=0$ which in the
case of compact manifolds is one.
We conclude that a 10-dimensional scalar field leads to a massless scalar field in 4-dimensions.
This procedure can be easily generalized to obtain the 
\todoen{Complete}

We begin the KK reduction of a $p$-form $C_p$ with the expansion 
\begin{equation}
  C_p=\sum_{k,q} c_q^k(x^m)\wedge C^k_{p-q}(x^\mu)
\end{equation}
Massless 4-dimensional $(p-q)$-form fields correspond to $dc_q=d^\dagger c_q=0$ so 
Since there is a single harmonic $q$-form in each $q$-cohomology class, 
the number of 4-dimensional massless $(p-q)$-forms is the dimension of the $H_q$ cohomology group.
\todoin{Complete KK. Specify for c1 and c3.}

The KK reduction of the 10-dimensional metric is applied considering its components separately:
\begin{itemize}
  \item The $G_{\mu\nu}$ components correspond to scalar fields in the compact space
    Thus, 10-dimensional graviton reduces to a 4-dimensional graviton.
  \item The $G_{\mu m}$ components would correspond to 4-dimensional vector bosons, associated
    to 6-dimensional vector fields in the compact space. 
    The masslessness condition of the 4-dimensional field would implies that the 6-dimensional 
    vectors are Killing vectors associated to continuous isometries of the compact space, which
    in the case of Calabi-Yau manifolds are non-existent.
    So the $G_{\mu m}$ components do not lead to any massless fields in 4 dimensions.
  \item The $G_{m n}$ components
    moduli of the internal space
    $h_{21}$ $h_{11}$
\end{itemize}

The closed string spectrum can be arranged into supermultiplets of 4-four dimensional $\mathcal N=2$,
which are:
A single supergravity multiplet, composed of a graviton $G_{\mu\nu}$, a gauge boson $A_\mu$ and two gravitinos $\psi$ with opposite chiralities. 
$h_{1,1,}$ vector multiplets, composed of a gauge boson $C_{i\bar j\mu}$ a
$h_{2,1}+1$ vector hypermultiplets
\todoin{How do we arrange the fields into multiplets?}

\section{Type IIA on orientifold projections}

\subsection{Generalities of orientifolds}
If we compactify a type II string theory on a Calabi-Yau manifold, we  obtain a four-dimensional
$\mathcal N=2$ supersymmetric theory.
This degree of supersymmetry does not allow for chiral fermions, so Calabi-Yau compactifications
of type II theories have no straightforward application in the context of model building.
An option to reduce the supersymmetry to $\mathcal N=1$  is to apply the orientifold
projection, which consists in modding out the action of $\Omega R$,
where $\Omega$ is the worldsheet parity, which makes strings become unoriented, and
$R$ is a particular $\mathbb Z_2$ symmetry of the compact six-dimensional space

In type IIA string theory we define $R=\mathcal R (-1)^{F_L}$.
$\mathcal R$ satisfies the condition that it is an involution (squares to the identity) and 
acts antiholomorphically on the complex coordinates of the internal space ($\mathcal R: z_i \to \bar z_i$).
This implies that the Kähler and the holomorphic three-form transform as $J\to -J$ and $\Omega_3 \to \bar \Omega_3$.

\todoin{(-1)F anticommutes with the fermionic coordinates of the string. (-1)FL only anticommutes with left moving spectrum? Where do we apply it to?}


\todoin{What points belong to a plane or another?}
The fixed points under $\mathcal R$ define the orientifold planes in the model and are denoted
as O$p$-planes, where $p$ is the spatial dimension.
They span the entire four-dimensional Minkowski space and wrap a compact $(p-3)$-cycle on the internal space.

O$p$-planes carry RR charge, so in order to eliminate RR tadpoles we must also introduce D$p$-branes,
which carry opposite charge.
It is important to note that D$p$-branes do not need to wrap the same 3-cycles as the O$p$-planes
to remove RR tadpoles.


\subsection{D-branes on Calabi-Yau manifolds}

\todoin{Can a submanifold of a CY not be a cycle?}
In order to obtain stable D6-brane configurations on a type IIA theory compactified on a Calabi-Yau manifold, 
we impose that they wrap around volume minimizing 3-cycles on the compact space, so energy is minimized too.
The volume minimizing condition means that the branes must wrap special Lagrangian 3-cycles in the internal space.
Special Lagrangian 3-cycles $\Pi$, are defined by
\begin{equation}
  k|_\Pi = 0 , \qquad \Im (e^{-i\phi}\Omega_3)|_\Pi=0
\end{equation}
for some real $\phi$, where $k$ is the Kähler two-form and $\Omega_3$ the holomorphic three-form.
The $e^{-i\phi}\Omega_3$ is referred as a calibration and the special Lagrangian is calibrated with respect to it.%, for every choice of $\phi$.
The volume of the special Lagrangian 3-cycle is
\begin{equation}
  \mathrm{Vol}(\Pi)=\int_\Pi \Re(e^{-i\phi}\Omega_3)
\end{equation}

D6-branes wrapped around a special Lagrangian cycle are guaranteed to preserve 4-dimensional $\mathcal N=1$ supersymmetry. 
This preserved supersymmetry coincides with that preserved by O$p$-planes only if $\phi=0$.
\todoin{Should relocate? What if no common susy with branes or Oplanes?}
The open string spectrum of stacks of $N_a$ D$6_a$-branes with multiplicities wrapping special Lagrangian 3-cycles $\Pi_a$ 
can be classified into two sectors: strings that stretch from one stack to itself and those that stretch between to different stacks, $6_a$ and $6_b$.

Strings that stretch over $6_a$ lead to $U(N_a)$ vector multiplets of 4-dimensional $\mathcal N=1$
$b_1(\Pi_a)$ chiral multiplets in the adjoint representation,
composed of the internal components of the gauge field along $\Pi_a$,
geometric moduli of the $3-cycle$ and their fermion superpartners.

  chiral fermion intersection number $I_{ab}=[\Pi_a][\Pi_b]$ which transform as $(\mathbf{N_a},\mathbf{\bar N_b})$ relative angle between.

\subsection{Orientifold compactifications with intersecting D-branes}

$z_i \to \bar z_i$

\begin{equation}
  \theta_1 + \theta_2 + \theta_3 = 0
\end{equation}

\begin{equation}
  \sum_a N_a [\Pi_a] + \sum_a N_{a} [\Pi_{a'}] - 4 [\Pi_{O6}]=0
\end{equation}

\subsection{Effective action of intersecting D-branes on Calabi-Yau orientifolds}

\begin{equation}
\begin{aligned}
C_1 = C_1(x)\\
B_2 = b_2(x)+b^p \omega_p \\
C_3 = A_1^P(x)\wedge \omega_P + C^K(x)\alpha_K -\tilde C_K(x)\beta^K
\end{aligned}
\end{equation}


Dimensional reduction

\begin{equation}
  S_{DBI} = -\mu_p \int_{D_p} e^{-\phi}\sqrt{\det (G+B-2\pi \alpha' F)}
\end{equation}

\begin{equation}
  S = \mu_p \int_{D_p} \sum_q \Tr e^{2\pi \alpha' F-B}+\cdots
\end{equation}

\begin{equation}
  \frac{1}{g^2}=e^{-\phi}\frac{(\alpha')^{(3-p)/2}}{(2\pi)^{p-2}}\mathrm{Vol}(\Pi_{p-3})
\end{equation}

\begin{equation}
  f_a = \frac{(\alpha')^{3/2}}{(2\pi)^4}\qty[e^{-\phi}\int_{\Pi_a}\Re(e^{-i\phi_a}\Omega_3)+ i\int_{\Pi_a}c_3]
\end{equation}
