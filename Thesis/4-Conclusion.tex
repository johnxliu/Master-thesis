\chapter{Conclusion and outlook}

Having model building on mind, we have considered SUSY special lagrangian three-cycles which arise as invariant directions under the orientifold involution and determined that they  are rigid (no deformations) and can be extended beyond the Fermat point.
Unfortunately, their intersection numbers do not generate chiral fermions, but we are working on the quintic (the simplest Calabi-Yau realised as an algebraic variety) and perhaps there are more intricate smooth Calabi-Yau background on which we can apply the techniques discussed here.


%We have seen that D$6$-branes wrapping SLags three-cycles on Fermat's quintic present a serious obstruction
%for model building, since they do allow a chiral spectrum.
We are able to evaluate qualitatively the effect of a complex structure deformation on the volume of a special lagrangian three-cycle on the quintic (for small deformations).
%Since the intersection number is a topological quantity, it is not expected to change under deformations.
%In contrast, the volume of SLags does change, but it requires to calculate an integral which we do not 
%now how to compute analytically.  
%Thus, we have only obtained an estimate of how the volume varies under a particular deformation.
It would be interesting to obtain a more precise value of the volume and to discuss how it is affected by other deformations.

We encountered that deformations of the Fermat's quintic lead to singularities.
In a specific example  we obtain a conifold singularity, but generic deformations would require a detailed study of the type of singularities, possibly in terms  of singular surfaces and curves on the quintic.  
