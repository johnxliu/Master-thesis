\chapter{Type IIA generalities}

The are six main types of strings theories: a bosonic string theory and five supersymmetric 
string theories, the latter include both bosons and fermions.
%While the bosonic string theory can be formulated solely using closed strings, superstring theories
%require closed and open strings.
In this thesis we will only work with the supersymmetric type IIA theory. 

\subsection{TIIA spectrum}
Type IIA string theory requires ten space-time dimensions to be consistent.
The flat 10 dimensional space-time bosinic spectrum of type IIA can be classified according to
the boundary conditions of the strings, whether we consider Ramond (R) or Neveu–Schwarz (NS) conditions.
In the NS-NS sector, we have the dilaton $\phi$, a two-form $B_2$ and a graviton $G_{\mu\nu}$,
while in the R-R sector we have the 1- and 3-forms $c_1, c_3$.
The fermions, which belong to the NS-R and R-NS sectors, are two opposite-chirality gravitinos $\psi$ and two opposite-chirality dilatinos $\lambda$.
%This is the perturbative spectrum around the vacuum, which is made of closed strings only. 

\subsection{D-brane introduction}

A generalization of strings are D$p$-branes, which are $p$-dimensional extended objects.
Thus, a D$1$-brane would correspond to a string, a D$2$-brane would be a membrane and so on.
The existence of D$p$-branes can be motivated, in the weak coupling limit, as 
objects where open strings end, so they are a way to impose Dirichlet boundary conditions on open strings.
In fact, D-branes should be though as new non-perturbative states in their own right.
\todoin{More subtle than this. In perturbative regime branes are fully described in terms of strings. f-strings, T and S dualities 

lead to a more democratic formulation.}

We can study the dynamics of a D$p$-branes in terms of the dynamics of open strings with endpoints
attached to the D$p$-brane.
Let us consider the open string excitations of a D$p$-brane, the latter spanning $p+1$ dimensions and transverse to $d-p-1$ dimensions.
The presence of the D$p$-brane breaks the ten-dimensional Poincaré invariance of the theory, because
particles propagate on the $(p+1)$-dimensional volume of the D$p$-brane only.
This implies that massless particles must transform under irreducible representations of $SO(p-2)$, instead of
$SO(d-2)$.
The massless spectrum in $(p+1)$ dimensions of the open string theory is composed by a gauge boson $A^\mu$ ($\mu=0,\ldots,p$),
$9-p$ real scalars $\phi^i$ and some fermions $\lambda_a$.
This particle content can be arranged into a vector supermultiplet of $U(1)$ with $16$ supersymmetries
in $(p+1)$ dimensions. 
Thus, a D$p$-brane reduces the degree of supersymmetry of the type IIA theory by half.

\todoin{Branes supported on sLag cycles represent BPS states}

In order to find out the action of a D$p$-brane we must realize that it corresponds to the
$(p+1)$-dimensional effective action of the massless open string excitations of the D$p$-brane.
Since a D$p$-brane breaks the translational symmetry of the vacuum, Goldstone bosons
vev of the scalar fields determine the position of the D$p$-brane, and the fluctuations of the scalar
fields determine the dynamics of the D$p$-brane.
The resulting action of the bosonic sector of the D$p$-brane is the sum of a Dirac-Born-Infeld term $S_{DBI}$ and a Chern-Simons term $S_{CS}$.
\todoin{Complete}

The DBI term carries the information of how a D$p$-brane interacts with the NSNS fields. 
It takes the form

\begin{equation}
  S_{DBI}= -\frac{\alpha'^{-(p+1)/2}}{(2\pi)^p}\int_{W_{p+1}}  d^{p+1}x f(\phi^i, A^\mu, G_{\mu\nu}, B_2)
\end{equation}

where the precise expression of $f$ is unimportant to us.
\todoin{What is the background here and what is dynamical? BG: closed string fields and Dynamical: open string fields?}



%\begin{equation}
%  S_{DBI}= -\frac{\alpha'^{-(p+1)/2}}{(2\pi)^p} \int d^{p+1} x e^{-\phi} \sqrt{-\det(P\qty[G+B]-2\pi \alpha'F)}
%\end{equation}
%
%gauge field strength $F=d $
%
%\begin{equation}
%  \mu_p = \frac{(\alpha')^{-(p+1)/2}}{(2\pi)^p}
%\end{equation}
%
%pullback

If we expand the DBI action in powers of $\alpha'$, we obtain the Yang-Mills term
\begin{equation}
  S_{YM} = \frac{\alpha'^{-(p-3)/2}}{4g_s (2\pi)^{p-2}} \int d^{p+1} x \sqrt {-g} \Tr F_{\mu\nu} F^{\mu\nu}
\end{equation}
which allows us to identify the Yang-Mills coupling as 

\begin{equation}
g^2_{YM} = g_s \alpha'^{(p-3)/2}(2\pi)^{p-2}
\end{equation}

The Chern-Simons term is topological in nature and describes how D$p$-branes interact with RR-fields.
%Which means that D$p$-branes are carry RR charge.

%\begin{equation}
%S_{CS}= \mu_p \int P[\sum_p c_q]\wedge e^{2\pi\alpha F_2 -B_2}\wedge \hat A (R)
%\end{equation}

It is convenient to generalize the single D$p$-brane configuration to $N$ parallel D$p$-branes.
In order to determine the spectrum of a stack of D$p$-branes, we consider open strings with endpoints 
attached to either a single brane or two of them.

In the case of $N$ coincident D$p$-branes, all configurations lead to massless states, so the 
gauge symmetry is increased from $U(1)$ to $U(N)$.
The massless spectrum is composed of $(p-1)$-dimensional $U(n)$ gauge bosons, $(9-p)$ real
scalars in the adjoint representation and several fermions in the adjoint representation.

If all branes are separated from each other, strings that stretch from a brane to itself correspond to massless gauge bosons that belong to $U(1)^N$ .
In contrast, strings that stretch from one brane to another lead to massive particles whose
mass increases with the distance between branes.
The lightest of these particles have opposite charges with respect to the $U(1)$ of each brane.
\todoin{Maybe mention orientation}

\subsection{Intersecting branes}
\todoin{Reorganize}
In our work, D6-branes are interesting because they are used as a basis to obtain 4d chiral fermions.
A basic configuration of two stacks of D6-branes that intersect over a 4d subspace of their volumes.

Strings that stretch from a coincident stack of $N$ D6-branes lead to 7d $U(N)$ gauge bosons, three real adjoint scalars and their fermion superpartners.

String that stretch from a stack of $N_1$ D6-branes to another stack $N_2$ are localized at the interesection,
in order to minimize their energy. 
They lead to a 4d fermion charged in the $(\mathbf{N_1},\mathbf{N_2})$ of $U(N_1)\times U(N_2)$.

\todoin{SUSY condition}

%\todoin{RR-sector -> RR-tadpole cancellation?}
%
%D6-branes acts as RR sources so 
%Cancellation of RR-tadpoles imply that the 4d effective theory is anomaly free.


\subsection{SUGRA}
 
The low-energy theory of the ten-dimensional type IIA string theory is type IIA SUGRA.
The spectrum of Type IIA SUGRA has as elementary fermions, which belong to the massless spectrum (NS-R and R-NS) of type IIA theory,
two Majora-Weyl gravitinos of the same chirality $\psi_M$ and two dilatinos $\lambda$. 
\todoin{Complete bosons}

32 supersymmetries $\mathcal N=(1,1)$ in ten dimensions

