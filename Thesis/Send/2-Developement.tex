\section{Type IIA compactifications}


STRING THEORY GENERALITIES


In the following section we motivate the requirement that additional dimensions are compactified
over a Calabi-Yau manifold (a compact complex manifold of $SU(3)$ holonomy).

Since Type IIA requires nine spatial dimensions but we only observe three, we need to compactify six of 
them over a small region.
We assume that the manifold $M$ is factorizable into a four-dimensional maximally symmetric space-time $T$ and a six-dimensional compact space $K$,
$M =  T\times K$.

Type IIA string theory on 10 dimensional flat space has a large degree of supersymmetry,
but the compactification choice can either preserve some degree of supersymmetry or remove it completely.
We will consider compactifications which leave some supersymmetry.
Our main reason for this choice is that they provide a nice way to obtain realistic particle 
physics models. 
In particular, we will see that a $\mathcal N=1$ supersymmetric theory allows for chiral fermions.
In addition, supersymmetric configurations are easier to study before tackling more general compactifications.

An approach to construct a supersymmetric theory is roughly to extend the Poincaré algebra into a super-Poincaré algebra
by adding supersymmetry generators which satisfy specific 
anti-commutation relations, instead of commutation relations. 

A conserved charge $Q$ associated to an unbroken supersymmetry annihilates the vacuum $\ket{\Omega}$,
so $Q\ket{\Omega}=0$.
This in turn means that for any operator $U$, $\ev{ \{Q,U\} }{\Omega}=0$.
If $U$ is a fermionic operator, we derive that the variation of the operator under the supersymmetry
transformation is $\delta U = \{ Q, U\}$.
Taking this as the classical limit, $\delta U = \ev{\delta U}{\Omega}$.
Thus, we conclude that at the classical level $\delta U =\ev{ \{Q,U\} }{\Omega}= 0$ for any fermionic field $U$.

The low-energy theory of the ten-dimensional type IIA string theory is type IIA SUGRA, which although it has
$\mathcal N=2$ instead of $\mathcal N=1$, it is possible to obtain $\mathcal N=1$ through orientifold projections. 
%The type IIA string theory has as NS-NS fields the dilaton $\phi$, the metric $G$ and the two-form field $B$.
%In the RR sector we have the one-form and three-form, $c_1$ and $c_3$.
\textcolor{red}{N=2 or N=(1,1). Something about the worldsheet.}

The spectrum of Type IIA SUGRA has as elementary fermions two Majora-Weyl gravitinos of the same chirality $\psi_M$ and two dilatinos $\lambda$. 
In the same way that a translation generated by the momentum operator is parametrized by a vector and
a rotation is a parametrized by an antisymmetric tensor, a supersymmetry transformation generated
by $Q_\alpha$ is parametrized 
by a spinor $\eta_\alpha$.
The variation of the gravitino field under a supersymmetry transformation is
\begin{equation}
  \delta \psi_M = D_M \eta + \mathrm{(fluxes)}
\end{equation}

Where $D_M$ is the covariant derivative on $M$.
Supersymmetry preservation means that all variations must be zero. 
We assume that all fluxes vanish.
This leads to the constraint that $\eta$ is a covariantly constant spinor
\begin{equation}
  \delta \psi_M = D_M \eta = 0
%  \delta \xi^a &= -\frac{1}{4g\sqrt \phi} \Gamma^{MN} F^a_{MN} \eta 
\end{equation}

%The equation implies that there exists a spinor $\eta$ such that  $[D_M,D_N]\eta=R_{MNPQ} \Gamma^{PQ} \eta=0$.
If we particularize this equation to the four-dimensional space-time  $T$, which is a maximally symmetric space, 
it imposes that $T$ is Minkowski space and thus, $\eta$ only depends on the compact coordinates.

%\todoin{\url{https://groups.google.com/forum/#!topic/sci.physics.research/rrBoIXk9Rw0}}

The existence of a single covariantly constant spinor on the compact manifold can be reformulated
as a topological condition, namely that the holonomy group (whose precise definition is given in the next chapter) of the compact manifold is $SU(3)$.
A compact manifold of $SU(3)$ (local) holonomy is the definition of a Calabi-Yau manifold.
The holonomy group being a proper subgroup of $SU(3)$ is equivalent to having more than one covariantly 
constant spinor, which would lead to a larger degree of supersymmetry preserved.
%%TODO: topological vs differential. group structure vs holonomy. existence vs cov. const.

The existence of a covariantly constant spinor implies  for a TIIA theory that there are two four-dimensional supersymmetry parameters and therefore, $\mathcal N =2$.

\section{Type IIA on Calabi-Yau manifolds}
We examine more closely what the existence of a covariantly constant spinor field implies on the compact space. 

Let us consider a Riemannian manifold $K$ of dimension six with a spin connection $\omega$, which 
is in general a $SO(6)$ gauge field.
If we parallel transport a field $\psi$ around a contractible closed curve $\gamma$, the field becomes
$\psi'=U\psi$ where $U=\mathcal P e^{\int_\gamma dx \omega}$ and $\mathcal P$ denotes the path ordering of 
the exponential.
The set of transformation matrices associated to all possible loops form the holonomy group of the manifold, 
which must be a subgroup of $SO(6)$.

%\todo{What does the existence of a non-trivial spinor tell us}
A covariantly constant spinor is left unchanged when parallel transported along a contractible
closed curve, so the holonomy matrices of a manifold that admits a covariantly constant spinor 
must satisfy $U\eta = \eta$.
Taking into account the Lie algebra isomorphism $\mathfrak{so}(6)\simeq \mathfrak{su}(4)$ we identify the positive
(negative)-chirality spinors of $SO(6)$ with the fundamental $\mathbf 4$ ($\mathbf {\bar 4}$)
of $SU(4)$.
Let us take that $\eta$ is a positive chirality spinor, so it transforms according with the 
$\mathbf 4$ of $SU(4)$.
In order to have a covariantly constant spinor, the holonomy group must be such that the $\mathbf 4$
representation decomposes into a singlet.
This decomposition is achieved if the holonomy group is $SU(3)$ so that
 \begin{align}
  SO(6)  &\to SU(3)\\
  \mathbf 4 &\to \mathbf 3 \oplus \mathbf 1
\end{align} 
%Then, $\eta$ is left invariant under $SU(3)$ transformations and can be written as
%\begin{equation}
%  \eta= 
%  \qty(
%  \begin{array}{c}
%    0\\
%    0\\
%    0\\
%    \eta_0
%  \end{array}
%  )
%\end{equation}
%In other words, the existence of a covariantly constant spinor implies that the holonomy group 
%of the manifold is $SU(3)$.


We can also check that the 2-form $\mathbf {15}$ and the 3-form $\mathbf{20}$ decompositions contain a singlet, 
$\mathbf {15}\to \mathbf 8\oplus \mathbf 3\oplus \bar {\mathbf 3}\oplus \mathbf 1$ and 
$\mathbf {20}\to \mathbf 6\oplus \bar{\mathbf 6}\oplus\mathbf 3\oplus \bar {\mathbf 3}\oplus \mathbf 1\oplus \mathbf 1$,
so they are globally well defined.
We refer to the 2-form as $J$ and the 3-form as the holomorphic three-form $\Omega$.
Raising an index of $J$ we obtain an almost-complex structure, which satisfies $(J^2)^i_j=-\delta^i_j$.
For a particular point of the manifold, we can form a basis of complex coordinates $z^i$ from the real coordinates $x^i$,
as $z^1=x^1+ix^2$, $z^2=x^3+ix^4$ and $z^3=x^5+ix^6$,
in which $J=idz^i\otimes dz^i - i d\bar z^{\bar i}\otimes d\bar z^{\bar i}$.
If we can extend this particular form of $J$ to the neighborhood of any point, $J$ is said to be integrable
and the manifold is complex.
An integrable almost-complex structure is referred to as a complex structure. 
The integrability condition is equivalent to the Nijenhuis tensor 
\begin{equation}
  N^k_{ij}= J^l_i(\partial_l J^k_j - \partial_j J^k_l) - J_j^l (\partial_l J^k_i - \partial_i J^k_l)
\end{equation}
vanishing everywhere.
Intuitively, a complex manifold can though as a manifold that can be covered by
complex coordinate charts which are related at their intersections by holomorphic transition functions.

It is useful to define with the aid of the metric the form $k=g_{i\bar j} dz^i \wedge d\bar z^{\bar j}$.
A manifold is Kähler if $dk=0$ and $k$ is then called the Kähler form.
It can be shown that the holonomy group being contained in $U(N)$ implies that the manifold is Kähler.

%The only  $U(3)$ invariants in the $\mathbf{6}$ representation of $SO(6)$ are the identity and
%$\bar I$.
%
%We can also form a tensor field on $K$ of the type $J^i_j(y)=g^{ik}(y) \bar\eta \Lambda_{kj} \eta(y)$.
%For each point $y$, we can consider $J^i_j$ as a matrix that acts on the tangent space, so $v^i \to J^i_j v^j$.
%In this sense, $J^i_j$ is a real, traceless and $SU(3)$ invariant matrix, which means that it must be proportional
%to $\bar I$.
%We had already seen that $\bar I = -I$, this an example of an almost-complex structure, which is 
%a tensor field $J$ that satisfies $J^2=-I$.
%
%If we employ complex coordinates, we can diagonalize $J$ so that the non-zero components are
%$J^a_b=i\delta^a_b$ and $J^{\bar a}_{\bar b}=-i\delta^{\bar a}_{\bar b}$. This particular choice
%is know as the canonical form.
%
%It is always possible to choose particular coordinates to bring $J$ to the canonical form at a particular 
%point.
%But in general, the canonical form will not hold at an open neighborhood of a point.
%If a manifold admits a set of coordinates (called local holomorphic coordinates) such that at every
%point, the canonical form holds for an open neighborhood, then the almost complex structure is integrable.
%
%The necessary and sufficient condition for integrability is that the Nijenhuis tensor
%
%\begin{equation}
%  N^k_{ij}= J^l_i(\partial_l J^k_j - \partial_j J^k_l) - J_j^l (\partial_l J^k_i - \partial_i J^k_l)
%\end{equation}
%
%vanishes.
%An integrable almost-complex structure is a complex structure and a manifold with a complex structure
%is a complex manifold.

%\todoin{Coordinate definition of complex manifold}

HOMOLOGY AND p-FORMS / deRHAM COHOMOLOGY

DOLBEAULT COHOMOLOGY / HODGE NUMBERS

INTERSECTION NUMBERS


MODULI SPACE
%
%
%\todoin{Calabi-Yau definition: SU(n) global holonomy<->non-vanishing n-form -> vanishing first Chern class <->vanishing Ricci curvature<->local 
%SU(n) holonomy. If simply connnected, both definitions are equivalent.}
%We start from a complex manifold with a metric st. it is a Calabi-Yau manifold.
%We then deform it's metric while it remains CY
%
%
%Given a Calabi-Yau manifold with a particular complex structure and metric, we could ask ourselves
%when the deformations

\section{Orientifold projections and D-branes}
\subsection{Orientifold planes and D-branes}
%
%\todoin{D-brane motivation through O-planes.}
%If we compactify a Type II string theory on a Calabi-Yau manifold, we  obtain a four-dimensional
%$\mathcal N=2$ supersymmetric theory.
%In order to allow for chirality, we must obtain a $N=1$ theory. This can be done through the orientifold
%projection, which consists in modding out $\Omega \mathcal R$ acting on the 10d manifold, 
%where $\Omega$ is the worldsheet parity and $\mathcal R$ is a particular $\mathbb Z_2$ symmetry.
%
%
%\todoin{Understand RR charges.}
%\begin{equation}
%  S= -\frac{1}{4\pi \alpha'}\int d^2\sigma \epsilon^{\alpha\beta} B_{\mu\nu}  \partial_\alpha X^\mu \partial_\beta X^\nu
%\end{equation}
%
%\subsection{D6-branes in flat 10d space}
%
%We have seen that if we compactify an heterotic string theory on a Calabi-Yau manifold, we obtain
%$\mathcal N=1$ which allows chirality. 
%This is not the case of Type II theories, which lead to $\mathcal N=2$.
%In order to reduce the degree of supersymmetry and thus obtain chiral 4d fermions, two D6-branes in flat 10d can intersect over a 4d region.
%Before considering D6-branes over a Calabi-Yau compactification, we consider the flat space
%case.
%
%The open string spectrum of an intersection of a stack of $N_1$ D6-branes and a stack of $N_2$ D6-branes
%can be classified as:
%
%\begin{itemize}
%  \item Strings stretching from one stack to itself, which lead to 7d $U(N_i)$ gauge bosons, three real
%    adjoint scalars and their fermion superpartners, .
%  \item String that go from one stack to the other are localized at the intersection. 
%    Their associated fields are charged under the bi-fundamental representation $(N_1, \bar N_2)$ of 
%    $U(N_1)\times U(N_2)$, which includes a 4d chiral fermion in $(N_1,\bar N_2)$.
%\end{itemize}
%
%\subsection{D6-branes on a torus}
%
%Let us consider type IIA theory compactified on a 6-torus $T^6=T^2 \times T^2 \times T^2$.
%
\subsection{D6-branes on a Calabi-Yau. Special Lagrangians.}
%In order to obtain stable D6-brane configurations, we impose that they wrap around volume 
%minimizing 3-cycles, which are special Lagrangian 3-cycles and satisfy
%\begin{equation}
%  J|_\Pi = 0 , \qquad \Im (e^{-i\phi}\Omega_3)|_\Pi=0
%\end{equation}
%
%The volume of the special Lagrangian 3-cycle is
%\begin{equation}
%  Vol(\Pi)=\int_\Pi \Re(e^{-i\phi}\Omega_3)
%\end{equation}
%
%\todoin{Spectrum}



%INTRODUCE NON-ABELIAN GAUGE BOSONS
%
\subsection{Model building}
