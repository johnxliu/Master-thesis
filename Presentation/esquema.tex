\documentclass[oneside,secnumarabic,10pt,nobalancelastpage,nofootinbib,a4paper]{article}

\usepackage[spanish, activeacute]{babel} %Definir idioma español
\usepackage[utf8]{inputenc} %Codificacion utf-8
\usepackage[T1]{fontenc}

\usepackage{graphics}      % standard graphics specifications
\usepackage{graphicx}      % alternative graphics specifications
\usepackage{url}           % for on-line citations
\usepackage{bm}            % special 'bold-math' package

\usepackage{amsmath,amssymb,amsfonts}
\usepackage{physics}
\usepackage[exponent-product=\cdot,range-phrase=--,range-units=single]{siunitx}


\usepackage{todonotes}
\usepackage{microtype}
\usepackage{xcolor}

\graphicspath{{images/}}

\newcommand{\ra}[1]{\renewcommand{\arraystretch}{#1}}
\nouppercaseheads

%Default inline comments
\newcommand{\todoin}[2][]{\todo[inline,#1]{#2}}

\usepackage{mathtools}
\DeclarePairedDelimiter\floor{\lfloor}{\rfloor}
\interfootnotelinepenalty=10000
\begin{document}

INTRO

La teoría de cuerdas es la principal candidata para describir de forma unificada las interacciones gauge
del modelo estándar con la gravitación de la relatividad general.
Desde el punto de vista fenomenológico, es interesante obtener el modelo estándar como una teoría
efectiva de la teoría de cuerdas a bajas energías.
\textcolor{red}{Existen diversas formas de intentar recuperar el modelo estándar a partir de la teoria de cuerdas.}
En este trabajo nos hemos centrado en un formalismo muy prometedor, que se basa en considerar intersecciones
de D6-branas en la teoría de cuerdas Tipo IIA.
\textcolor{red}{Estas D6-branas han de enrollarse sobre subvariedades muy particulares, denominadas lagrangianos especiales.}
El objetivo de este trabajo ha consistido en estudiar unas características especialmente interesantes de los lagrangianos especiales:
su topología y posibles deformaciones, sus intersecciones y sus volúmenes.
Elegimos como espacio de la compactificación un caso ampliamente estudiado, la quíntica aplicando la proyección orientifold.

--------------

COMPACTIFICACION

La teoría de cuerdas requiere 10 dimensiones para que sea consistente, pero solo observamos 4 dimensiones en la práctica.
\textcolor{red}{Por tanto, las dimensiones adicionales han de estar compactificadas en una variedad con un volumen extremadamente pequeño.}
Entonces, el espacio se decompone en un espacio-tiempo de Minkowski de cuatro dimensiones y un espacio interno de seis dimensiones.
\textcolor{red}{La variedad que describe este espacio interno no esta determinada a priori por la teoria.
Sin embargo, desde el punto de vista fenomenológico, hay variedades especialmente interesantes.
En concreto, es interesante que la compactificación preserve parcialmente la supersimetría con la que cuenta
la teoría de cuerdas.}
La famila de variedades que cumplen esta propiedad son las variedades de Calabi-Yau.

Una variedad de Calabi-Yau es una variedad equipada con una estructura compleja, una forma de Kähler, una 
tres forma y una metrica, cuyo tensor de Ricci es nulo. 
\textcolor{red}{No todos estos objetos son independientes, para una estructura compleja y una forma de Kähler dadas, existe 
una única métrica con tensor de Ricci nulo.
Por otra lado, la estructura compleja y la forma de Kähler no son únicas, si no que existe todo un espacio 
formado por todas sus deformaciones continuas posibles, el espacio modular.}

Un ejemplo muy estudiado de variedad de Calabi-Yau es la quíntica de Fermat, la cual es una subvariedad del
espacio proyectivo complejo en cuatro dimensiones.
El espacio proyecto complejo de cuatro dimensiones se define como el espacio complejo de cinco dimensiones sin
el origen en el que se identifican los puntos proporcionales.
Es decir módulo la relación de equivalencia entre puntos proporcionales.

Una vez definido el espacio proyectivo, la quíntica de Fermat es la hipersuperficie que satisface la siguiente ecuación 
quíntica.

--------------

\textcolor{red}{Las compactificaciones de una teoria de cuerdas de tipo II en una variedad Calabi-Yau contienen demasiada supersimetría 
como para obtener fermiones quirales en la teoría efectiva cuadridimensional, por lo que su utilidad es limitada.}
La solución a esta limitación consiste en tomar la aplicar la proyección orientifold, que consiste en tomar el espacio cociente
de la variedad de Calabi-Yau módulo la acción del operador $\Omega$ y de $\mathcal R$.
$\Omega$ es un operador que invierte la orientación de las cuerdas,
mientras que $\mathcal R$ es una simetría $\mathbb Z_$.
En la quíntica, tomamos como acción de $\mathcal R$ la conjugación compleja de las coordenadas.

El efecto de la proyección orientifold sobre el espacio modular es que para las deformaciones 
del este tipo, donde añadimos un monomio multiplicado por un parámetro psi, psi ha de ser real.

--------------

BRANAS

En teoria de supercuerdas de tipo IIA tenemos como objetos fundamentales las cuerdas abiertas y las Dp-branas.
Las Dp-branas pueden verse como una generalización de las cuerdas a objetos de $(p+1)$-dimensiones.
De este modo, una D1-brana sería una D-cuerda, una D2-brana sería una membrana tridimensional y así sucesivamente.
Consideraremos únicamente D6-branas, que ocupan todo el espacio de Minkowski y que se enrollan sobre tres dimensiones internas.

Sobre las D6-branas pueden acabar las cuerdas abiertas.  
Las excitaciones de cuerdas abiertas con extremos en D6-branas describen partículas en la teoría efectiva cuadridimensional.
En concreto, las cuerdas con extremos en $N$ D6-branas apiladas tienen en su espectro no massivo bosones gauge del grupo $U(N)$.
En cambio, las cuerdas con un extremo en una pila de $N_1$ D6-branas y otro en $N_2$ D6-branas que se intersecan, 
tienden a localizarse en la intersección para minimizar la energía.
Estas describen como partículas no masivas fermiones que se transforman en la representación fundamental de $U(N_1)$ y antifundamental de
$U(N_2)$.

--------------

Para que las D-branas describan una configuración estable, su tensión ha de ser mínima. 
Por tanto, han de enrollarse sobre una subvariedad cuyo volumen sea mínimo. 
Las subvariedades que cumplen esta condición se denominan lagrangianos especiales y 
verifican que sobre ellos la forma de Kähler se anula y la forma holomorfa toma valores puramente reales, salvo por una fase.
En la quíntica bajo la proyección orientifold, los puntos de la quíntica reales forma una clase de lagrangianos especiales.
Estos coinciden justamente con los planos orientifold.
Existen otros lagrangianos especiales que se obtienen mediante rotaciones adicionales.

Se puede determinar que la topología de los lagrangianos especiales que hemos obtenido en la quintica es la del espacio real proyectivo de dimensión
tres. Esto a su vez implica que los lagrangianos especiales no pueden ser deformados de manera continua. 
Por tanto, las D-branas se mantienen rígidas y el grupo gauge asociado a una pila de branas no se ve roto espontáneamente debido a que las 
branas se separen.

--------------

Por otro lado, obtenemos que no hay ninguna interesección entre lagrangianos especiales que preserve la supersimetría.
Esto significa que la quíntica no permite describir fermiones quirales. 
Una solución que algunos autores proponen es añadir un sector oculto de branas que no interseque a las branas del modelo estándar.

Además, es interesante calcular el volumen de los lagrangianos especiales, pues determinan la constante de acoplo de la 
teoría gauge de las D-branas situadas que enrollan el lagrangiano especial. Concretamente, el cuadrado de la constante de acoplo
es inversamente proporcional al volumen del lagrangiano especial.

La determinación del volumen en la quíntica presenta una dificultad, pues no sabemos cómo resolver analíticamente la integral de volumen tridimensional.
Si restringimos el dominio de integración a las coordenadas positivas únicamente, obtenemos una expresión analítica para la cota inferior 
del volumen. Su valor numérico es aproximadamente 0.61.

--------------

Una vez estudida la quíntica de Fermat, tratamos sus deformaciones.
Hay 101 un deformaciones posibles, pero solo consideramos la siguiente, donde añadimos el producto de todas las coordenadas.

Determinamos que la propiedades topológicas permanecen invariantes: los lagrangianos especiales siguen siendo rígidos y no hay interesecciones
entre ellos.

En cambio, si observamos una variación del volumen que depende de $\psi$. 
Si restringimos el dominio de integración a valores positivos, podemos dividir el volumen obtenido entre el volumen que hemos calculado previamente
sin deformaciones.
Observamos que para valores, pequeños de $\psi$, el volumen decrece predominantemente de forma lineal.
Esto se traduce en la constante de acoplo aumenta al aplicar la deformación.

--------------

Como comentario final, hemos visto que los lagrangianos supersimétricos en la quíntica son rígidos,
lo cual es conveniente desde el punto de vista fenomenológico.
Como contrapartida, no hay interseciones entre lagrangianos especiales, por lo que no obtenemos fermiones
quirales.

Aun así, sería conveniente aplicar un tratamiento similar a otras compactificaciones, donde puede que sí sea posible
obtener intersecciones.

Finalmente, sería interesante estudiar el resto de deformaciones de la quíntica y hallar un método para el cálculo preciso del volumen.

\end{document}
